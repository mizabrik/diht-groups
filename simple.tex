\documentclass[main]{subfiles}

\begin{document}

\section{Простые группы}
\begin{definition}
  Группа $G$ называется \emph{простой}, если в ней ровно две нормальных
  подгруппы: $G$ и $\{ e \}$.
\end{definition}
\begin{remark}
  Группа из одного элемента не является простой.
\end{remark}

Пусть $G$ "--- произвольная группа,
$G = G_0 \NGS G_1 \NGS \dots \NGS G_n = \{ e \}$ "---
субнормальный ряд подгрупп ($G_i \ne G_{i + 1}$).
Если $\Factor{G_i}[G_{i + 1}]$ "--- не простая,
то существует нетривиальная
$\Factor{H}[G_{i + 1}] \NSG \Factor{G_i}[G_{i + 1}]$,
тогда $H$ можно вставить между $G_i$ и $G_{i + 1}$,
ибо $G_i \NGS H \NGS G_{i + 1}$.
Если эта процедура закончится
(в частности, это так для всех конечных групп),
то получим субнормальный
ряд с простыми факторам.

\begin{remark}
  Полученный субнормальный ряд называется \emph{композиционным рядом}
  группы $G$.
\end{remark}

Для любых двух композиционных рядов группы $G$ наборы факторов
совпадают с точностью до перестановки и изоморфизма
(теорема Жордана-Гёльдера).

\begin{proposition}
  Абелева группа проста $\oTTo G \cong \Integer_p$ при простом $p$.
\end{proposition}
\begin{proof}
  Пусть $G$ "--- абелева простая группа, $e \ne g \in G$.
  Тогда $\langle g \rangle \NSG G \To \langle g \rangle = G$.
  Значит, $G \cong \Integer$ или $G \cong \Integer_n$.
  Если $G \cong \Integer$, то $\Integer \NGS 2\Integer$ "--- $G$ не проста.
  Пусть $G \cong \Integer_n$ и $n$ "--- составное, т. е. $n = kl$,
  $k, l > 1$.
  Тогда $\Integer_n \NGS k \Integer_n \ne \Integer_n
  \To G$ непроста.

  Если $G \cong \Integer_p$,
  то $\Forall{H < G} |H| \divides p$,
  т. е. $|H| = p$ или $|H| = 1$,
  а значит $H = G$ или $H = \{ e \}$ "--- $G$ проста.
\end{proof}

\begin{theorem}
  Группа $A_5$ проста.
\end{theorem}
\begin{lemma}
  Пусть $H \NSG G$, $|G:H| = 2$, $h \in H$.
  Тогда, если $C_G(h) \ne C_H(h)$, то $h^H = h^G$.
  В противном случае $h^G = h^H \cup h_1^H$ и $|h^H|
  = |h_1^H|$ для некоторого $h_1 \in H^G$.
\end{lemma}
\begin{proof}
  Напоминание: $C_G(h) = \{ g \in G \mid hg = gh \}$,
  $|h^G| = |G : C_G(h)|$.

  Пусть существует $g \in C_G(h) \setminus C_H(h) = C_G(h) \setminus H$.
  Тогда $G = H \cup gH = H \cup Hg$. Значит, $h^G = h^H \cup h^{gH} =
  h^H \cup (h^g)^H = h^H$, т. к. $g \in C_G(h)$.

  Пусть $h^G = h^H$. Тогда $\Forall{g \in G \setminus H}
  h^g = h^x$, $x \in H \To h^{gx^{-1}} = h \To gx^{-1} \in C_G(h) \setminus H$.
  В этом случае $C_G(h) \ne C_H(h)$.
  Значит, если $C_G(h) = C_H(h)$, то $h^G \ne h^H$, но $h^G =
  h^H \cup (h^g)^H$, где $g \in G \setminus H$. Обозначим
  $h_1 = h^g \To h^G = h^G \sqcup h_1^H$. Наконец,
  $h_1^H = h^{gH} = h^{Hg}$, тогда биекция между
  $h^H$ и $h_1^H$ задаётся очень просто: $h^H \ni x \mapsto x^g \in h^{Hg}$.
  Итак, $|h^H| = |h_1^H|$.
\end{proof}
\begin{proof}[Доказательство теоремы]
  Пусть $H \NSG A_5$, $H \ne \{ e \}$.
  Тогда $H$ есть объединение нескольких
  классов сопряжённости в $A_5$.
  Пусть $\sigma = \Cycle{a_1 & \dots & a_k} \in S_n$.
  Тогда $\sigma^{\tau^{-1}} = \tau \sigma \tau^{-1} =
  \Cycle{\tau(a_1) & \dots & \tau(a_k)}$.
  
  Значит, классы сопряжённости 
  (и их мощности)
  элементов $A_5$ в $S_5$
  и в \( A_5 \) таковы:

  \begin{center}
    \begin{tabularx}{0.82\textwidth}{|l|X|}
      \hline
      в $S_5$ & в $A_5$ \\
      \hline
      \hline
      $e^{S_5} = \{ e \}$  & \(|\{ e \}| = 1\) \\
      \hline
      $\Cycle{1&2&3}^{S_5} = \{ \Cycle{i&j&k} \}
       $ & $\Cycle{4&5} \in C_{S_5}(\Cycle{1&2&3}) \setminus A_5$,

      то есть
      $|\Cycle{1&2&3}^{A_5}| = |\{ \Cycle{i&j&k} \}| = 20$ \\
      \hline
      $(\Cycle{1&2}\Cycle{3&4})^{S_5} =
      \{ \Cycle{i&j}\Cycle{k&l} \}$ &
      $\Cycle{3&4} \in C_{S_5}
      (\Cycle{1&2}\Cycle{3&4}) \setminus
      A_5$,

      то есть
      $|(\Cycle{1&2}\Cycle{3&4})^{A_5}| =
      |\{ \Cycle{i&j}\Cycle{k&l} \}| = 15$ \\
      \hline
      $\Cycle{1&2&3&4&5}^{S_5} = \{ \Cycle{i&j&k&l&m}
      \}$  &
      \( 24 \ndivides |A_5| = 60 \To \) 
      два класса:

      $|\Cycle{1&2&3&4&5}^{A_5}| = 12$, $|\Cycle{1&2&3&5&4}^{A_5}| = 12$ \\
      \hline
    \end{tabularx}
  \end{center}

  Из тех чисел $1, 20, 15, 12, 12$ нельзя составить нетривиальную сумму,
  делящую $|A_5| = 60$ и содержащую $1$.
  Значит, раз $|H| \divides 60$
  и $e \in H$,
  $|H| = 60$ и $H = A_5$.
\end{proof}

\begin{theorem}
  При $n \ge 5$ группа $A_n$ проста.
\end{theorem}
\begin{remark}
  $A_4 \NGS V_4$.
\end{remark}

\begin{proof}
  Уже знаем: \( A_n = \GenGroup{ \{ \Cycle{i & j & k} \} } \),
  \( A_n \) действует на \( \{ 1, \dots, n \} \)
  и \( \St(i) \cong A_{n - 1} \).

  Индукция по $n \ge 5$, база уже доказана.
  Пусть теперь $n \ge 6$, $\{ e \} \ne H \NSG A_n$.

  Докажем, что существует нетривиальная
  перестановка \( e \ne \sigma \in H \) такая,
  что \( \Exists{i} \sigma \in \St(i) \).
  Рассмотрим $e \ne \tau \in H$. Б. о. о.
  $\tau = \Cycle{1 & 2 & \dots}\dots$ "--- разложение $\tau$ в произведение
  независимых циклов. $\tau \in A_n \To \tau$ нетривиально переставляет
  хотя бы $3$ элемента.
  Значит, $\Exists{k} \tau(k) \notin \{ 1, 2, k \}$
  (возможно, $k \in \{ 1 ,2 \}$).
  Пусть $\tau(k) = l$. Наконец,
  пусть $p, q \notin \{ 1, 2, k, l \}$, $p \ne q$, $p, q \in \{1, \dots, n \}$.
  Обозначим $\tau_1 = \tau^{(l,p,q)}$,
  тогда $\tau_1(1) = 2$, $\tau_1(k) = q$.
  Значит,
  $(\tau_1 \tau^{-1})(2) = 2$,
  $(\tau_1 \tau^{-1})(l) = q \ne l$,
  т. е.  $e \ne \tau_1 \tau^{-1} \in H \cap \St(2)$.

  Пусть $e \ne \sigma \in H \cap \St(n)$. Тогда $H \cap \St(n)
  \NSG \St(n) \cong A_{n - 1}$. Значит,
  $H \cap \St(n) = \St(n)$ по предположению индукции.
  В частности, $\Cycle{1&2&3} \in H \To \Cycle{1&2&3}^{A_n} \subseteq H$,
  но $\Cycle{1&2&3}^{A_n} = \{ \Cycle{i&j&k} \}$. Т. о. $H > \langle
  \Cycle{1&2&3}^{A_n} \rangle = A_n$.
\end{proof}

\begin{remark}
  $A_5$ "--- неабелева простая группа наименьшего возможного порядка.
\end{remark}

\begin{remark}
  Пусть $F$ "--- поле.
  Тогда простой является группа
  \[
    PSL_N(F) = \Factor{SL_n(F)}[\Centre{SL_N(F)}],
  \]
  где $\Centre{SL_n(F)} =
  \{ \lambda E \mid \lambda \in F, \lambda^n = 1 \}$,
  если
  \begin{enumerate}\renewcommand{\labelenumi}{\asbuk{enumi})}
    \item $n \ge 3$ 
    \item $n = 2$ и $|F| \ge 4$.
  \end{enumerate}
\end{remark}

\begin{exercise}
  $PSL_2(F_2) \cong S_3$,
  $PSL_2(F_3) \cong A_4$,
  %"<Со звёздочкой">:
  $PSL_2(F_4) \cong PSL_2(F_5) \cong A_5$.
\end{exercise}

\begin{theorem}
  Группа $SO_3$ проста, где $SO_3 = O_3 \cap SL_3(\Real)$.
\end{theorem}
\begin{remark}
  $SO_3$ "--- группа \emph{вращений} трёхмерного евклидового
  пространства. Действительно, если $A \in SO_3$, то у $A$
  есть с. з. $1$ и $A$ реализует вращение вокруг соответствующего
  собственного вектора.
\end{remark}
\begin{proof}
  Как выглядят классы сопряжённости в $SO_3$? Пусть
  $g, h \in SO_3$, и пусть $h$ "--- вращение вокруг $l$
  на угол $\alpha$.
  Тогда $g h g^{-1}$ "--- вращение на угол $\alpha$ вокруг $g(l)$,
  поскольку $h(x) = y \To ghg^{-1}(gx) = g(y)$.
  Значит, $h^{SO_3}$ состоит из всех вращений на угол $\alpha$.
  Пусть $\{ e \} \ne H \NSG SO_3$. Пусть $e \ne h \in H$ --
  пусть это вращение на $\alpha$ относительно $l$. Тогда в $H$ содержатся
  все вращения на $\alpha$. Пусть $l(\phi)$ "--- прямая, образующая
  угол $\phi$ с $l$, $x(\phi)$ "--- вращение вокруг $l(\phi)$ на $\alpha$.
  Тогда $x(\phi) \in H$ (считаем, что $x(\phi)$ непрерывно меняется
  при изменении $\phi$). Положим $y(\phi) = h(x(\phi))^{-1}$. Тогда
  $y(0) = id$, а $y(\phi)$ "--- вращение на некоторый угол $\beta(\phi)$,
  $\beta(0) = 0$. $\beta(\phi)$ "--- также непрерывная функция
  аргумента $\phi$ ($\beta$ выражается через $\tr y(\phi)$),
  $\beta$ "--- не тождественный ноль. Значит, значения
  $\beta(\phi)$ заметают некоторый интервал
  $[0, \alpha_0]$, $\alpha_0 > 0$, т. к. $y(\phi) \in H$,
  в $H$ лежат все вращения на углы $\gamma \in [0, \alpha_0]$
  $\To$ в $H$ лежат все вращения 
\end{proof}

\begin{remark}
  $SO_n$ проста при $n = 3$ и $n \ge 5$.
\end{remark}
\end{document}

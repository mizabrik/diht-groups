\documentclass[main]{subfiles}

\begin{document}
% --- 2016-09-01 ---
\section{Основные понятия}
\begin{definition}
  \emph{Группа} "--- это непустое множество $G$
  с бинарной операцией $\cdot$,
  обладающей следующими свойствами:
  \begin{itemize}
    \item Ассоциативность:
      $a \cdot (b \cdot c) = (a \cdot b) \cdot c$
    \item Существование нейтрального элемента:
      $\Exists{e \in G} \Forall{a \in G}
      ae = ea = a$
    \item Существование обратного элемента:
      $\Forall{a \in g} \Exists{a^{-1} \in G}
      aa^{-1} = a^{-1}a = e$
  \end{itemize}
  Обозначение "--- $(G, \cdot)$,
  если операция очевидна "--- просто $G$.
  Группа называется \emph{абелевой},
  если операция \( \cdot \) коммутативна
  ($a \cdot b = b \cdot a$).
\end{definition}
\begin{remark}
  Нейтральный и обратные элементы единственны.
\end{remark}

\begin{definition}
  \emph{Подгруппа} $H < G$ "---
  это непустое подмножество $H \subseteq G$,
  замкнутое относительно операций:
  \( \Forall{a, b \in H} a \cdot b \in H \),
  \( \Forall{a \in H} a^{-1} \in H \).
\end{definition}
\begin{remark}
  $H$ "--- также группа, с той же операцией
  (ограниченной на \( H \)).
\end{remark}

\begin{definition}
  \emph{Порядок группы} "--- число её элементов $|G|$.
  \emph{Порядок элемента} группы $g \in G$ "---
  это наименьшее $n \in \Natural$ такое,
  что $g^n = e$ (и $\infty$, если такого $n$ нет).
  Обозначение: $|g|$ или $\ord g$.
\end{definition}

\begin{definition}
  Если $M \subset G$, то \emph{подгруппа, порождённая $M$} "---
  это пересечение всех подгрупп, содержащих $M$.
  Также $\langle M \rangle = \{ a_1 \dots a_n \mid
  a_i \in M \lor a_i^{-1} \in M \}$.
  Обозначение: $\langle M \rangle $.
  Если существует \( g \in G \) такой, что
  \( \langle g \rangle = G \),
  то группа \( G \) "--- \emph{циклическая}.
\end{definition}
\begin{example}
  $\langle G \rangle = G$, $\langle \emptyset \rangle = \{ e \}$.
\end{example}
\begin{remark}
  $\ord g = |\langle g \rangle|$.
\end{remark}

\begin{definition}
  Биекция $\phi: G \to H$, сохраняющая операцию
  ($\phi(g_1 g_2) = \phi(g_1) \phi(g_2)$),
  называется \emph{изоморфизмом групп} \( G \) и \( H \).
  Если он существует, то $G$ и $H$ \emph{изоморфны}
  (\( G \cong H \)).
\end{definition}

\section{Примеры групп}
\begin{enumerate}
  \item $(\Integer, +)$, $(\Integer_n, +)$ "---
    единственные (с точностью до изоморфизма)
    циклические группы.
    Подгруппа циклической группы "--- также циклическая.
  \item $(F, +)$, $(F^*, \cdot)$, где $F$ "--- поле.
  \item $(V, +)$, где $V$ "--- линейное пространство.
  \item $S_n$ "--- группа перестановок $n$ элементов
    (т. е. биекций
    $\{ 1, \dots, n \} \to \{ 1, \dots, n \}$)
    относительно композиции.
    Перестановку можно записать в виде таблицы,
    или же в виде произведение
    независимых циклов
    (цикл $\pi = \Cycle{a_1 & \dots & a_k}$ "--- это перестановка такая,
    что $\pi(a_i) = a_{i + 1}$ для $i = 1, \dots, k - 1$
    и $\pi(a_k) = a_1$, остальные
    элементы неподвижны).
    Кроме того, $S_n$ порождается множеством всех транспозиций.
    Знак перестановки $\sigma \in S_n$ есть
    $(-1)^\sigma = \sgn \sigma = (-1)^{N(\sigma)}$,
    где $N(\sigma)$ "--- число инверсий в $\sigma$
    (совпадает по чётности
    с количеством транспозиций
    в любом разложении $\sigma$).
  \item $GL_n(F)$ "--- группа невырожденных матриц 
    над $F$ относительно умножения.
  \item $GL(V)$, где $V$ "--- линейное пространство над $F$,
    "--- обратимые преобразования
    $V$ относительно композиции.
    $GL(V) \cong GL_{\dim V}(F)$.
  \item Подгруппы этих групп, в частности:
    \begin{itemize}
      \item $A_n < S_n$ "--- подгруппа всех чётных перестановок.
      \item $SL_n(F) < GL_n(F)$ "--- подгруппа всех матриц с единичным определителем.
      \item $O_n < GL_n(\Real)$ "--- подгруппа всех ортогональных матриц.
      \item $\Complex_n < \Complex^*$:
	$\Complex_n = \{ z \in \Complex \mid z^n = 1 \}$,
	$\Complex_n \cong \Integer_n$.
    \end{itemize}
\end{enumerate}

\section{Смежные классы}
\begin{definition}
  Пусть $H < G$, $g \in G$.
  \emph{Левый смежный класс} элемента $g$ по $H$ "--- это $gH$,
  \emph{правый} "--- $Hg$,
  где $AB = \{ ab \mid a \in A, b \in B \}$
  для \( A, B \subset G \)
  (вместо одного элемента подразумевается
  множество из этого элемента).
  $\Factor{G}[H]$ "--- множество всех левых смежных классов по $H$,
  $\Factor[H]{G}$ "--- правых.
\end{definition}
\begin{remark}
  Для любых $a, b \in G$
  $aH \cap bH \ne \emptyset
  \oTTo b^{-1}a \in H
  \oTTo aH = bH
  \oTTo b \in aH$.

  Значит, левые (правые) смежные классы "--- разбиение $G$.
\end{remark}

\begin{proposition}
  Пусть $H < G$. Тогда $\Factor{G}[H]$ равномощно $\Factor[H]{G}$.
\end{proposition}
\begin{proof}
  Построим биекцию $\phi : \Factor{G}[H] \to \Factor[H]{G}$:
  $\phi(gH) = H g^{-1}$.
  Заметим, что $\phi(gH) = Hg^{-1} = H^{-1} g^{-1} = (gH)^{-1}$,
  а тогда $\phi$ корректно определено
  и является отображением из $\Factor{G}[H]$ в $\Factor[H]{G}$.
  Биективность следует из существования
  \(\phi^{-1} : Hg \mapsto g^{-1}H \).
\end{proof}
\begin{remark}
  Отображение $gH \mapsto Hg$ не всегда корректно определено.
\end{remark}

\begin{definition}
  Если $H < G$, то \emph{индексом} $H$ в $G$
  называется $|G:H| = |\Factor{G}[H]| = |\Factor[H]{G}|$.
\end{definition}

\begin{theorem}[Лагранжа]
  Для конечной группы
  $|G| = |H| \cdot |G:H|$.
\end{theorem}
\begin{corollary}
  $|H|$ делит $|G|$, и для любого \( g \in G \) $|g|$ делит $|G|$.
\end{corollary}
\end{document}

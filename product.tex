\documentclass[main]{subfiles}

\begin{document}

\section{Прямое произведение групп}
\begin{definition}
  Пусть \( G_1, G_2 \) "--- группы.
  Их (внешним) \emph{прямым произведением}
  называется множество \( G = G_1 \times G_2 =
  \{ (g_1, g_2) \mid g_1 \in G_1, g_2 \in G_2 \} \)
  с операцией \( (g_1, g_2) \cdot (g_1', g_2') =
  (g_1 g_1', g_2 g_2') \).
\end{definition}

\begin{proposition}
  Если \( G_1 \), \( G_2 \) "--- группы,
  то \( G_1 \times G_2 \) "--- группа.
\end{proposition}
\begin{proof}
  Ассоциативность следует из ассоциативности \( G_1 \)
  и \( G_2 \), нейтральным будет \( (e_1, e_2) \) а
  обратным к \( (g_1, g_2) \) "--- \( (g_1^{-1}, g_2^{-1}) \).
\end{proof}

\begin{proposition}
  Для групп \( G_1 \), \( G_2 \), \( G_3 \) верно:
  \begin{enumerate}
    \item \( G_1 \cong G_1 \times \{ e \} \NSG G_1 \times G_2 \)
    \item \( G_1 \times G_2 \cong G_2 \times G_1 \)
    \item \( (G_1 \times G_2) \times G_3 \cong
      G_1 \times (G_2 \times G_3) \)
  \end{enumerate}
\end{proposition}
\begin{proof}~
  \begin{enumerate}
    \item Очевидно, \( G_1 \times \{ e \} < G_1 \times G_2 \).
      При этом,
      \[
	(g_1', g_2')^{-1} (g_1, e) (g_1', g_2') =
	(g_1'^{-1} g_1 g_1', e) \in G_1 \times \{ e \}
	\To
	G_1 \times \{ e \} \NSG G_1 \times G_2.
      \]
      Изоморфизм же осуществляется отображением
      \( \phi : g \mapsto (g, e) \).
    \item Изоморфизм осуществляется отображением
      \( (g_1, g_2) \mapsto (g_2, g_1) \).
    \item Изоморфизм "---
      \( ((g_1, g_2), g_3) \mapsto (g_1, (g_2, g_3)) \).
      \qedhere
  \end{enumerate}
\end{proof}

\begin{remark}
  Таким образом можно определить
  \( G_1 \times \dots \times G_k \).
  Более того, аналогично можно определить
  и  прямое произведение бесконечного
  числа групп: \( \prod_{\alpha \in I} G_\alpha \),
  где \( I \) "--- произвольное множество индексов.
\end{remark}

% --- 2016-10-06 ---
\begin{theorem}
  Пусть \( A, B \NSG G \), \( AB = G \),
  \( A \cap B = \{ e \} \).
  Тогда \( G \cong A \times B \).
\end{theorem}
\begin{proof}
  Покажем, что \( \Forall{a \in A, b \in B} ab = ba \).
  Действительно, рассмотрим
  \( a b a^{-1} b^{-1} \):
  \( a b a^{-1} \in B \) и \( b a^{-1} b^{-1} \in A \)
  из их нормальности,
  а тогда \( a b a^{-1} b^{-1} \in A \cap B = \{ e \} \),
  т. е. \( ab a^{-1} b^{-1} = e \To
  ab = ba \).

  Построим отображение
  \( \phi : A \times B \to G \):
  \( \phi : (a, b) \mapsto a b \).
  Тогда \( \phi((a_1, b_1) \cdot (a_2, b_2)) =
  \phi((a_1 a_2, b_1 b_2)) = a_1 a_2 b_1 b_2 =
  a_1 b_1 a_2 b_2 = \phi((a_1, b_1)) \cdot \phi((a_2, b_2)) \To \)
  \( \phi \) "--- гомоморфизм.
  Т. к. \( AB = G \), \( \Img \phi = AB = G \).
  Наконец, если \( (a, b) \in \Ker \phi \),
  то \( ab = e \To a = b^{-1} \in A \cap B = \{ e \}
  \To a = b = e \To \Ker \phi = \{ (e, e) \} \).
  Значит, \( \phi \) "--- изоморфизм.
\end{proof}

\begin{definition}
  В ситуации, описанной в теореме,
  \( G \) является
  \emph{внутренним прямым произведением}
  \( A \) и \( B \).
\end{definition}

\begin{remark}
  \( \Factor{G}[A] = \Factor{AB}[A] \cong
  \Factor{B}[B \cap A] = \Factor{B}[\{ e \}] \cong B \)
  по первой теореме об изоморфизме.
  Аналогично, \( \Factor{G}[B] = A \).
\end{remark}

\begin{proposition}
  Пусть \( G = A \times B \) "--- прямое произведение
  групп \( A \) и \( B \).
  Пусть \( A_1 \NSG A \), \( B_1 \NSG B \).
  Тогда \( A_1 \times B_1 \NSG A \times B \),
  причём \( \Factor{(A \times B)}[(A_1 \times B_1)] =
  (\Factor{A}[A_1]) \times (\Factor{B}[B_1]) \).
\end{proposition}
\begin{proof}
  Пусть \( \overline{A} = \Factor{A}[A_1] \),
  \( \overline{B} = \Factor{B}[B_1] \), и
  пусть \( \pi_A : A \to \overline{A} \),
  \( \pi_B : B \to \overline{B} \) "---
  соответствующие канонические эпиморфизмы.
  Рассмотрим \( \pi = \pi_A \times \pi_B :
  A \times B \to \overline{A} \times \overline{B} \),
  \( \pi(a, b) = (\pi_A(a), \pi_B(b)) \).
  Нетрудно видеть, что \( \pi \) "---
  эпиморфизм групп.
  \( \Ker \pi = \Ker \pi_A \times \Ker \pi_B =
  A_1 \times B_1 \).
  Итак, \( \overline{A} \times \overline{B} =
  \Img \pi \cong \Factor{(A \times B)}[\Ker \pi] =
  \Factor{(A \times B)}[(A_1 \times B_1)] \)
  (и \( A_1 \times B_1 = \Ker \pi \NSG G \)).
\end{proof}

В заключение, рассмотрим более общую ситуацию.
Пусть \( A \NSG G \), \( B < G \),
причём \( AB = G \), \( A \cap B = \{ e \} \).
В этом случае, \( G \) называется
полупрямым произведением \( A \) и \( B \):
\( G = A \leftthreetimes B \).

\begin{examples}~
  \begin{enumerate}
    \item \( S_n = A_n \leftthreetimes
      \langle (1, 2) \rangle \), \( n \ge 2 \).
      (но \( S_n \ncong A_n \times \langle (1, 2) \rangle \)
      при \( n \ge 3 \), ибо \(
      Z(S_n) = \{ e \} \)).
    \item \( S_4 = V_4 \leftthreetimes S_3 \).
  \end{enumerate}
\end{examples}

Как описать полупрямые произведения групп?
Пусть \( G = A \leftthreetimes B \),
\( \Forall{b \in B} b A b^{-1} = A \).
Значит, группа \( B \) действует сопряжением
на \( A \), т. е. возникает гомоморфизм
\( \psi : B \to \Aut A \):
\( [\psi(b)](a) = b a b^{-1} = a^{b^{-1}} \).
Задание групп \( A \), \( B \) и гомоморфизма
\( \psi \) однозначно задаёт \( G \).
Действительно, для любого \( g \in G \)
существует единственное разложение
\( g = ab \), \( a \in A \), \( b \in B \).
Умножение задаётся так:
\[
  (a_1, b_1)(a_2, b_2) =
  a_1 b_1 a_2 b_2 = a_1 (b_1 a_2 b_1^{-1}) b_1 b_2 =
  \underbrace{a_1}_{\in A} \underbrace{[\psi(b_1)](a_2)}_{\in A}
  \underbrace{b_1 b_2}_{\in B}
\]

\begin{exercise}
  Пусть \( A \), \( B \) "--- группы,
  \( \psi : B \to \Aut A \) "--- гомоморфизм.
  Тогда можно определить группу так: \( G = A \times B \),
  \( (a_1, b_1)(a_2, b_2) = (a_1 \cdot [\psi(b_1)](a_2), b_1 b_2) \).
\end{exercise}

\end{document}

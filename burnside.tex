\documentclass[main]{subfiles}

\begin{document}
\section{Лемма Бернсайда}

\begin{theorem}
  Пусть конечная группа \( G \) действует
  на \( \Omega \) транзитивно.
  Определим для \( g \in G \)
  \( F(g) = |\{ \omega \in \Omega \mid
  g(\omega) = \omega \}| \).
  Тогда
  \[
    \sum_{g \in G} F(g) = |G|
  \]
\end{theorem}
%\begin{remark}
%  \( |G(w)| = |\Omega| \).
%\end{remark}
\begin{proof}
  Занумеруем \( G = \{ g_i \}_1^n \) и
  \( \Omega = \{ \omega_i \}_1^k \),
  обозначим
  \[
    (i, j) = \begin{cases*}
      1, & если \( g_i(w_j) = w_j \) \\
      0, & иначе
    \end{cases*},
  \]
  тогда
  \begin{gather}
    F(g_i) = \sum_{j = 1}^k (i, j), \\
    \sum_{g \in G} F(g) = \sum_{i,j} (i, j).
  \end{gather}
  Заметим, что
  \[
    \sum_{i = 1}^n (i, j) = |\St(w_j)| =
    \frac{|G|}{|G(w_j)|} = \frac{|G|}{|\Omega|},
  \]
  а тогда сумма всех \( (i, j) \) "--- это
  \( \frac{|G|}{|\Omega|} \cdot |\Omega| = |G| \).
\end{proof}

Если \( G \) действует на \( \Omega \),
то можем определить действие на \( \Omega^2 \):
\( g( (\omega_1, \omega_2) ) = (g(\omega_1), g(\omega_2)) \).
\begin{definition}
  Действие \( G \) на \( \Omega \)
  называется \emph{2-транзитивным},
  если действие \( G \) на \( \Omega^2 \)
  транзитивно.
\end{definition}
\begin{exercise}
  Если действие 2-транзитивно, то
  \( \sum_{g \in G} F(G)^2 = 2|G| \).
\end{exercise}

\begin{corollary}[Лемма Бернсайда]
  Пусть конечная группа \( G \)
  действует на конечное \( \Omega \),
  обозначим за \( \Factor{\Omega}[G] \)
  множество орбит этого действия.
  Тогда
  \[
    |\Factor{\Omega}[G]| =
    \frac{1}{|G|} \sum_{g \in G} F(g).
  \]
\end{corollary}
\begin{proof}
  Пусть \( \Factor{\Omega}[G] = \{ \Omega_i \}_{i = 1}^k \),
  тогда тем же действием \( G \) действует на \( \Omega_i \)
  транзитивно. Обозначим
  \( F_i(g) = |\{ w \in \Omega_i \mid g(w) = w \}| \).
  Тогда по теореме \( \sum_{g \in G} F_i(G) = |G| \),
  при этом \( F(G) = \sum_{i = 1}^k F_i(G) \), а тогда
  \[
    \sum_{g \in G} F(g) =
    \sum_{i = 1}^k \sum_{g \in G} F_i(g) =
    k \cdot |G| = |\Factor{\Omega}[G]| \cdot |G|. \qedhere
  \]
\end{proof}

\begin{example}
  Пусть \( p \) "--- нечётное простое число,
  \( k \in \Natural \). Нужно найти количество
  ожерелий из \( p \) бусин, которые могут иметь
  \( k \) разных цветов, повороты и перевороты
  отождествляются.
  Если бы они не отождествлялись, ответом было бы \( k^p \).
  Обозначим множество фиксированных ожерелий
  за \( \Omega \),
  на него действует группа диэдра \( D_p \),
  тогда искомое число "--- \( |\Factor{\Omega}[D_p]| \).
  Орбита "--- одно ожерелье с точностью
  до поворотов и переворотов.
  По лемме Бернсайда, достаточно
  найти \( F(g) \) для любого \( g \in D_p \).
  \begin{enumerate}
    \item \( g = e \), \( F(g) = |\Omega| = k^p \).
    \item \( g \) "--- осевая симметрия,
      \( F(g) = k^{(p + 1)/2} \).
    \item \( g \) "--- нетривиальный поворот.
      Тогда он сохранит элемент только если все бусины
      имеют один цвет, т. к. \( p \) "--- простое,
      т. е. \( F(g) = k \).
  \end{enumerate}

  Итого, \( |\Factor{\Omega}[D_p]| =
  \frac{1}{2p} (k^p +k^{(p+1)/2} \cdot p + k(p - 1)) \).
\end{example}

\end{document}

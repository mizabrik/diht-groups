\documentclass[main]{subfiles}

\begin{document}

% 2016-11-17
Во время работы с абелевыми группами
мы считаем, что операция "--- это "<\( + \)">,
а вместо \( a^n \) пишем \( na \).

Мы уже встречали следующие абелевы группы:
\begin{enumerate}
  \item Циклические: \( \Integer \) или \( \Integer_n \)
    (и изоморфные им).
  \item \( \Integer^l 
    \times \Integer_{n_1} \times \dots \times \Integer_{n_k} \) "---
    также конечно порождена
    (она порождена \( k + l \) элементами вида
    \( (0, \dots, 0, 1, 0, \dots, 0 ) \)).
\end{enumerate}

Наша цель "--- доказать, что все конечно порождённые абелевы
группы изоморфны таким.

\begin{remark}
  Если \( \GCD(n, k) = 1 \),
  то \( \Integer_n \times \Integer_k \cong \Integer_{nk} \).
  Действительно, элемент \( (1, 1) \in \Integer_n \times \Integer_k \)
  имеет порядок \( nk \To \Integer_n \times \Integer_k =
  \langle (1, 1) \rangle \cong \Integer_{nk} \).
  Значит, наше представление не единственно.
\end{remark}
\begin{remark}
  Условие конечно порождённости существенно.
  Например, группа \( (\Rational, +) \)
  не представима в выписанном виде.
  Более того, она неразложима в нетривиальное
  прямое произведение, т. к. в ней нет двух
  нетривиальных подгрупп, пересекающихся по
  \( \{ 0 \} \).
\end{remark}

\section{Конечно порождённые абелевы группы без кручения}

\begin{definition}
  Пусть \( A \) "--- абелева группа.
  Её \emph{периодической частью} (или \emph{кручением})
  называется
  \[
    T(A) = \{ a \in A \mid \ord a < \infty \}.
  \]
  Группа \( A \) называется \emph{абелевой группой без кручения},
  если \( T(A) =  \{ 0 \} \).
\end{definition}

\begin{proposition}
  Если \( A \) "--- абелева группа, то \( T(A) < A \).
\end{proposition}
\begin{proof}
  \( 0 \in T(A) \To T(A) \ne \emptyset \).
  Если \( a, b \in T(A) \), то \( \Exists{n, k \in \Natural}
  na = kb = 0 \).
  Тогда \( n \cdot (-a) = -na = 0 \To -a \in T(A) \) и
  \( nk (a + b) = nka + nkb = k \cdot 0 + n \cdot 0 = 0
  \To a + b \in T(A) \).
  Значит, \( T(A) < A \).
\end{proof}
\begin{remark}
  Периодическая часть неабелевой группы не обязательно подгруппа.
  Скажем, в \( O_2 \) все осевые симметрии имеют порядок 2,
  но их произведение может быть поворотом бесконечного порядка.
\end{remark}

Пусть \( A \) "--- конечно порождённая абелева группа без кручения.

\begin{definition}
  Пусть \( a_1, \dots, a_n \in A \).
  Система элементов \( a_1, \dots, a_n \) называется
  \emph{независимой}, если
  \[
    \Forall{k_1, \dots, k_n \in \Integer}
    \sum_{i = 1}^n k_i a_i = 0 \To k_1 = \dots = k_n = 0.
  \]
  Эта система называется \emph{базисом} группы \( A \),
  если она независима
  и \( \langle a_1, \dots, a_n \rangle = A \).
\end{definition}
\begin{remark}
  Если \( a_1, \dots, a_n \) "--- базис в \( A \),
  то
  \( \Forall{b \in A}
  \ExistsOne{k_1, \dots, k_n \in \Integer}
  b = \sum_{i = 1}^n k_i a_i
  \)
\end{remark}

\begin{lemma}
  Пусть \( A = \langle a_1, \dots, a_n \rangle \),
  и \( b_1, \dots, b_k \in A \), \( k > n \).
  Тогда система \( b_1, \dots, b_k \) "--- зависима.
\end{lemma}
\begin{proof}
  Поскольку \( b_i \in \langle a_1, \dots, a_n \rangle \),
  \( (b_1, \dots, b_k) = ( a_1, \dots, a_n) S \),
  где \( S \in \Matrices{n}{k}[\Integer] \subseteq
  \Matrices{n}{k}[\Rational] \).
  Так как \( k > n \), столбцы \( S \) линейно зависимы
  над \( Q \), т. е.
  существует \( 0 \ne x' \in \Matrices{k}{1}[\Rational] \)
  такой, что \( S x' = 0 \).
  Домножив \( x' \) на произведение знаменателей элементов
  из \( x' \), получим \( x \in \Matrices{k}{1}[\Integer] \).
  Значит, \( (b_1, \dots, b_k) x = (a_1, \dots, a_n)Sx = 0 \).
  Т. к. \( x \ne 0 \), \( (b_1, \dots, b_k) \) "--- зависимы.
\end{proof}

\begin{theorem}
  Пусть \( A \) "--- конечно порождённая абелева группа без кручения,
  тогда в \( A \) есть базис.
  Более того, любые два базиса в \( A \) равномощны.
\end{theorem}
\begin{proof}
  Предположим противное.
  Тогда любая порождающая группу система зависима.
  Из всех конечных порождающих систем выберем
  систему из наименьшего числа элементов
  \( a_1, \dots, a_n \).
  Пусть \( s_1, \dots, s_n \) "--- коэффициенты зависимости:
  \( \sum_{i = 1}^n s_i a_i = 0 \), не все \( s_i \) "--- нули.
  Из всех таких систем \( a_1, \dots, a_n, s_1, \dots, s_n \)
  выберем такую, в которой \( 0 \ne |s_1| \) минимален.

  \begin{enumerate}
    \item Можно считать, что \( s_1 > 0 \)
      (иначе домножим \( s_i \) на \( -1 \)).
    \item Если \( s_1 = 1 \), то
      \( a_1 + \sum_{i = 2}^n s_i a_i = 0 \To
      a_1 = -\sum_{i = 2}^n s_i a_i \To
      A = \langle a_2, \dots, a_n \rangle \),
      противоречие.
    \item \( s_1 > 1 \). Пусть \( s_1 \ndivides s_i \)
      при некотором \( i \ge 2 \).
      Тогда \( s_i = q s_1 + r \), \( q \in \Integer \),
      \( r \in \Natural \)
      и \( 0 < r < s_1 \).
      Значит, \( 0 = \sum_{j = 1}^n s_j a_j =
      s_1 a_1 + (q s_1 + r) a_i +
      \sum_{2 \le j \le n, j \ne i} s_j a_j =
      s_1(a_1 + q a_i) + r a_i  +
      \sum_{2 \le j \le n, j \ne i} s_j a_j \).
      Заметим, что \( A = \langle a_1 + q a_i, a_2, \dots, a_n \rangle \),
      т. к. \( a_1 = (a_1 + q a_i) - q a_i \).
      Для новой системы порождающих есть зависимость,
      в которой встречается коэффициент \( 0 \ne |r| < |s_1| \) "---
      противоречие с выбором системы.
    \item Итак, \( s_1 > 1 \), \( s_1 \divides s_i \).
      Тогда \( 0 = \sum_{i = 1}^n s_i a_i =
      s_1 (\sum_{i = 1}^n \frac{s_i}{s_1} a_i) \To
      \sum_{i = 1}^n \frac{s_i}{s_1} a_i \in T(A) \To
      \sum_{i = 1}^n \frac{s_i}{s_1} a_i = 0 \To
      a_1 = -\sum_{i = 2}^n \frac{s_i}{s_1} a_i \).
      Противоречие.
  \end{enumerate}

  Осталось показать, что любые два базиса равномощны.
  Пусть \( (a_1, \dots, a_n) \) и \( (b_1, \dots, b_k) \)
  базисы в \( A \). Тогда каждая из этих систем
  порождает \( A \) и, по лемме,
  \( k \le n \le k \To k = n \).
\end{proof}

\begin{remark}
  Пусть \( a_1, \dots, a_n \) и \( b_1, \dots, b_n \) "---
  два базиса в \( A \).
  Тогда \( (a_1, \dots, a_n) = (b_1, \dots, b_n) S \)
  и \( (b_1, \dots, b_n) = (a_1, \dots, a_n) T \),
  \( S, T \in \Matrices{n}{n}[\Integer] \).
  Значит, \( (a_1, \dots, a_n) = (a_1, \dots, a_n)TS \).
  Т. к. выражение через базис единственно, \( TS = E
  \To \det T \cdot \det S = 1 \), а поскольку
  их определители также целочисленны,
  \( \det S = \det T = \pm 1 \).
  Наоборот, если \( a_1, \dots, a_n \) "--- базис,
  \( S \in \Matrices{n}{n}[\Integer] \), \( \det S = \pm 1 \),
  то \( (a_1, \dots, a_n) S \) "--- тоже базис, т. к.
  \( S^{-1} \in \Matrices{n}{n}[\Integer] \) по формуле Крамера.
\end{remark}

\begin{remark}
  В условиях нашей теоремы
  если \( a_1, \dots, a_n \) "--- базис в \( A \),
  то
  \[
    A = \langle a_1, \dots, a_n \rangle =
    \langle a_1 \rangle \times \dots \times \langle a_n \rangle \cong
    \Integer \times \dots \times \Integer = \Integer^n
  \]
\end{remark}

% "--- 2016-11-24 "---

\begin{definition}
  Пусть \( A = \langle a_1, \dots, a_k \rangle \) "--- абелева группа.
  \( A \) называется \emph{свободной абелевой группой} со свободными
  порождающими \( a_1, \dots, a_k \), если для любой абелевой
  группы \( B \) и элементов \( b_1, \dots, b_k \in B \)
  существует гомоморфизм \( \phi : A \to B \) такой, что
  \( \phi(a_i) = b_i \), \( i = 1, \dots, k \).
\end{definition}

\begin{remark}
  Две свободные абелевы группы с одним и тем же количеством порождающих
  изоморфны "--- аналогично обычным.
\end{remark}
\begin{remark}
  В качестве свободной абелевой группы с \( k \) порождающими можно взять
  \[
    A = \GenGroup{a_1, \dots, a_k}[[a_i, a_j] = e, 1 \le i < j \le k] =
    \Factor{F_k}[\GenNGroup{[f_i,f_j] \mid 1 \le i < j \le k}] =
    \Factor{F_k}[F_k']
  \]
\end{remark}

\begin{theorem}
  Свободная абелева группа с \( k \) свободными порождающими "--- это
  \( \Integer^k \) (с точностью до изоморфизма).
\end{theorem}
\begin{proof}
  Пусть \( A = \Integer^k \), положим
  \( a_i = ( 0, \dots, 0, 1, 0, \dots, 0) \).
  Тогда \( A = \langle a_1, \dots, a_k \rangle \),
  т.~к.~\( (x_1, \dots, x_k) = \sum_{i=1}^k x_i a_i \).
  Кроме того, для любой абелевой группы \( B \) и
  для любых \( b_1, \dots, b_k \in B \) можно определить
  \( \phi((x_1, \dots, x_k)) = \sum_{i = 1}^k x_i b_i \).
  Тогда \( \phi : A \to B \) "--- гомоморфизм, и
  \( \phi(a_i) = b_i \).
\end{proof}

\section{Строение конечно порождённых абелевых групп}
\begin{corollary}
  Пусть \( A = \GenGroup{a_1, \dots, a_k} \) "---
  конечно порождённая абелева группа.
  Тогда существует \( B \NSG \Integer^k \)
  такая, что \( A \cong \Factor{\Integer^k}[B] \).
\end{corollary}
\begin{proof}
  Пусть \( c_1, \dots, c_k \) "--- свободные порождающие группы
  \( \Integer_k \). Тогда существует гомоморфизм
  \( \phi : \Integer^k \to A \) такой, что \( \phi(c_i) = a_i \).
  Значит, \( \Img \phi = \langle \phi(c_1), \dots, \phi(c_k) \rangle = A \).
  Если \( B = \Ker \phi \), то \( A \cong \Factor{\Integer^k}[B] \) по
  основной теореме о гомоморфизмах.
\end{proof}

Итак, для описания конечно порождённых абелевых групп
полезно исследовать подгруппы в \( \Integer^k \).

\begin{theorem}
  Пусть \( A \) "--- свободная абелева группа, \( B < A \).
  Тогда \( B \) "--- также свободная абелева группа;
  причём в \( A \) и \( B \) существуют базисы
  \( a_1, \dots, a_k \) и \( b_1, \dots, b_l \) такие, что
  \( k \ge l \), \( b_i = m_i a_i \), \( m_i \in \Natural \),
  и \( m_1 \divides m_2 \divides \dots \divides m_l \).
\end{theorem}
\begin{proof}
  Пусть \( A \cong \Integer^k \). Индукция по \( k \).

  \paragraph{База} Если \( k = 1 \), то \( A \cong \Integer \), а
  \( B \cong n \Integer \), \( n \in \Integer_+ \).
  Тогда можно положить \( a_1 = 1 \) и
  \( b_1 = n \) (если \( n > 0 \)) или \( l = 0 \)
  (если \( n = 0 \)).

  \paragraph{Переход} Пусть \( k > 1 \). Если \( B = 0 \),
  то утверждение верно (при \( l = 0 \)). Пусть теперь
  \( B \ne 0 \). Для любого базиса \( a_1, \dots, a_n \) в \( A \)
  и для любого \( b \in B \setminus \{ 0 \} \) существуют целые
  \( n_i \) такие, что \( b = \sum_{i = 1}^k n_i a_i \).
  Выберем базис \( (a_1, \dots, a_k) \) в \( A \) и
  \( 0 \ne b_1 \in B \) так, что \( n_1 > 0 \) и \( n_1 \) "---
  наименьшее возможное.

  \begin{enumerate}
    \item Пусть \( n_1 \ndivides n_i \) при некотором \( i \ge 2 \).
      Тогда \( n_i = qn_1 + r \), \( q \in \Integer \),
      \( 0 < r \le n_1 - 1 \). Тогда
      \[
	b_1 = n_1 a_1 + n_i a_i + \sum_{j \ge 2, j \ne i} n_j a_j =
	n_1(a_1 + q a_i) + r a_i + \sum_{j \ge 2, j \ne i} n_j a_j.
      \]
    Значит,  в базисе \( (a_1 + q a_i, a_2, \dots, a_k) \) разложение
      \( b_1 \) содержит коэффициент \( r < n_1 \) "--- противоречие
      с выбором. Значит, \( n_1 \divides n_i \), \( i \ge 2 \).
      Положим \( a'_1 = \sum_{i = 1}^n \frac{n_i}{n_1} a_i =
      a_1 + \sum_{i \ge 2} \frac{n_i}{n_1} a_i \). Тогда
      \( (a_1', a_2, \dots, a_k) \) "--- базис в \( A \), причём
      \( b_1 = n_1 a_1' \).  Дальше будем считать, что \( a_1' = a_1 \).

    \item Пусть \( b \in B \), \( b = \sum_{i = 1}^k d_i a_i \).
      Предположим, что \( n_1 \ndivides d_1 \), \( d_1 = q n_1 + r\),
      \( o < r < n \). Значит, \( b - q b_1 = \sum_{i = 1}^k (d_i - q n_i)a_i =
      r a_1 + \sum_{i = 2}^k (d_i - q n_i) a_i \).
      Итак, в разложении элемента \( b - qb_1 \in B \) по базису
      \( (a_1, \dots, a_k) \) есть коэффициент с \( r < n_1 \) "---
      противоречит с выбором \( b_1 \).
      Т. к. \( b_1 = n_1 a_1 \), то \( n_2 = n_3 = \dots = n_k = 0 \).
      Предположим, что \( n_1 \ndivides d_i \)
      при некотором \( i \ge 2 \).
      Положим \( b_1' = b - \frac{d_1}{n_1} b_1 + b_1 \in B \),
      \( b_1' = n_1 a_1 + \sum_{i = 2}^k d_i a_i \). Тогда
      \( b_1' \) выражается через \( (a_1, \dots, a_k) \) с одним
      из коэффициентов равным \( n_1 \),
      применяя к нему старое рассуждение,
      получаем противоречие.
      Итак, \( n_1 \divides d_i, i = 1, \dots, k \). Это означает,
      что \( B < n_1 A \).

    \item Заметим, что \( A = \langle a_1, \dots, a_k \rangle =
      \langle a_1 \rangle \oplus \langle a_2, \dots, a_k \rangle \).
      Обозначим \( A^* = \langle a_2, \dots, a_k \rangle \) и
      положим \( B^* = B \cap A^* \),
      тогда
      \( B = \langle b_1 \rangle \oplus B^* \).
      Действительно,
      \(
	\Forall{b \in B} b = \sum_{i = 1}^k d_i a_i,
      \)
      и \( n_1 \divides d_i \).
      Тогда \( b = d_1 a_1 + \sum_{i = 2}^k d_i a_i = d_1 a_1 +  b^*  \), где
      \( d_1 a_1 = \frac{d_1}{n_1} b_1 \), а \( b^* = b - \frac{d_1}{n_1} b_1 \in B \),
      и \( b^* \in A^* \To b^* \in B^* \). Итак, \( \langle b_1 \rangle + B^* = B \).
      Кроме того, \( \langle b_1 \rangle \cap B^* \subseteq
      \langle a_1 \rangle \cap A^* = 0 \). Значит, эта сумма "--- прямая.
      Применим предположение индукции к \( B^* < A^* \).
      Получим согласованные базисы
      \( (a_2', \dots, a_k') \) в \( A^* \) и
      \( (b_2, \dots, b_l) \) в \( B^* \) такие, что
      \( b_i = m_i a_i' \), \( m_2 \divides \dots \divides m_l \).
      Тогда \( A = \langle a_1 \rangle \oplus \langle a_2', \dots, a_k' \rangle \To
      (a_1, a_2', \dots, a_k') \) "--- базис в \( A \).
      \( B = \langle b_1 \rangle \oplus \langle b_2, \dots, b_l \rangle \),
      причём \( (b_2, \dots, b_l) \) "--- базис в \( B^* \), а тогда
      \( (b_1, \dots, b_l) \) "--- базис в \( B \).
      Осталось выяснить, что \( n_1 \divides m_2 \).

      Это так, поскольку \( B < n_1 A \To \) коэффициенты разложения \( b_2 \)
      по базису \( a_1, a_2', \dots, a_k' \) делятся на \( n_1 \),
      т. е. \( n_1 \divides m_2 \).
      \qedhere
  \end{enumerate}
\end{proof}

\begin{corollary}
  Пусть \( C \) "--- конечно порождённая абелева группа. Тогда
  \[
    C \cong \Integer^t \times \Integer_{m_1} \times
  \dots \times \Integer_{m_l},
  \]
  где \(t \ge 0 \)
  и \( m_i \) "--- натуральные числа,
  \( m_i > 1 \),
  причём \( m_1 \divides m_2 \divides \dots \divides m_l \).
\end{corollary}
\begin{proof}
  Как мы знаем, \( C \cong \Factor{A}[B] \), где
  \( A \) "--- свободная абелева группа и \( B < A \).
  Выберем в \( A \) и \( B \) согласованные базисы (из теоремы).
  Тогда \( A = \langle a_1 \rangle \times \dots \times
  \langle a_k \rangle \), и \( B = \langle b_1 \rangle \times \dots \times
  \langle b_l \rangle \), при этом \( \langle b_i \rangle < \langle a_i \rangle \).
  Значит,
  \[
    \Factor{A}[B] \cong \Factor{\langle a_1 \rangle}[\langle b_1 \rangle]
    \times \dots \times \Factor{\langle a_l \rangle}[\langle b_l \rangle]
    \times \langle a_{l+1} \rangle \times \dots \times \langle a_k \rangle
    \cong \Factor{\Integer}[m_1 \Integer]
    \times \dots \times \Factor{\Integer}[m_l \Integer]
    \times \Integer \times \dots \times \Integer \cong
    \Integer^{k - l} \times \Integer_{m_1} \times \dots \times \Integer_{m_l}.
  \]
  Наконец, если \( m_i = 1 \), то \( \Integer_{m_i} = \{ 0 \} \) и этот
  сомножитель можно выкинуть.
\end{proof}
\begin{corollary}
  Пусть \( C \) "--- конечно порждённая абелева группа.
  Тогда
  \begin{equation}\label{abel-structure} \tag{*}
    C \cong \Integer^t \times \Integer_{p_1^{\alpha_1}} \times
    \dots \times \Integer_{p_s^{\alpha_s}}
  \end{equation}
  где \( p_1, \dots, p_s \) "--- простые (не обязательно различные),
  а \( \alpha_i \in \Natural \).
\end{corollary}
\begin{proof}
  Если \( m_i = p_1^{\alpha_1} \dots p_d^{\alpha_d} \),
  то \( \Integer_{m_i} \cong \Integer_{p_1^{\alpha_1}} \times
  \dots \times \Integer_{p_d^{\alpha_d}} \).
  Осталось применить это к каждому сомножителю
  вида \( \Integer_{m_i} \).
\end{proof}

Следующая цель "--- доказать единственность такого разложения
(точнее, единственность набора из \( t \) и системы
\( (p_1^{\alpha_1}, \dots, p_s^{\alpha_s}) \) "--- с точностью
до перестановки).

\begin{proposition}
  Пусть \( C \) "--- группа вида \eqref{abel-structure}.
  Тогда \( T(C) = \{ 0 \}^t \times \Integer_{p_1^{\alpha_1}}
  \times \dots \times \Integer_{p_s^{\alpha_s}} \) и
  \( \Factor{C}[T(C)] \cong \Integer^t \).
\end{proposition}
\begin{proof}
  Пусть \( x \in C \), \( x = (x_1, \dots, x_t, y_1, \dots, y_s) \).
  Если \( x_i \ne 0 \), то \( \ord x = \infty \), ибо
  \( \Forall{n \in \Integer_+} n x_i \ne 0 \).
  Если \( x_1 = \dots = x_t = 0 \), то
  \( p_1^{\alpha_1} \dots p_s^{\alpha_s} x = 0 \).
  Итак, \( T(C) \) охарактеризован.

  Тогда \( \Factor{C}[T(C)] \cong
  (\Factor{\Integer}[\{ 0 \}])^t \times
  \Factor{\Integer_{p_1}^{\alpha_1}}[\Integer_{p_1}^{\alpha_1}]
  \times \dots \times
  \Factor{\Integer_{p_s}^{\alpha_s}}[\Integer_{p_s}^{\alpha_s}]
  \cong \Integer^t \).
\end{proof}

\begin{theorem}
  Пусть \( F \) "--- поле, а \( G < F^* \),
  \( |F| < \infty \). Тогда \( G \) "--- циклическая.
  В частности, если \( |F| < \infty \), то
  \( F^* \) "--- циклическая.
\end{theorem}
\begin{proof}
  Поскольку \( |G| < \infty \), \( G \) "--- конечно порождённая
  абелева группа \( \To G \cong \Integer_{m_1} \times
  \dots \times \Integer_{m_k} \), где
  \( 1 < m_1 \divides m_2 \divides \dots \divides m_k \).
  Тогда \( \Forall{g \in G} g^{m_k} = 1 \),
  т. е. все элементы \( G \) "--- это корни уравнения
  \( x^{m_k} - 1 = 0 \To \) их не более \( m_k \),
  т. е. \( |G| \le m_k \). Значит, \( k = 1 \)
  и \( G \cong \Integer_{m_k} \).
\end{proof}

\end{document}

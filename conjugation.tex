\documentclass[main]{subfiles}

\begin{document}
% "--- 2016-09-09 ---

\section{Сопряжение}
\begin{definition}
  Пусть $G$ "--- группа, $g, x \in G$.
  Тогда элементом, \emph{сопряжённым} к $g$ при
  помощи $x$,
  называется $g^x = x^{-1} g x$.
\end{definition}
\begin{example}
  Две матрицы одного преобразования
  в различных базисах
  сопряжены в $GL_n(F)$,
  их сопрягает матрица перехода.
\end{example}

\begin{proposition}~
  \begin{enumerate}
    \item $g^{xy} = (g^x)^y$
    \item $(g_1 g_2)^x = g_1^x g_2^x$
    \item $(g^{-1})^x = (g^x)^{-1}$
  \end{enumerate}
\end{proposition}
\begin{proof}~
  \begin{enumerate}
    \item $g^{xy} = y^{-1} x^{-1} g x y = y^{-1} g^x y = (g^x)^y$.
    \item $g_1^x g_2^x = x^{-1} g_1 x x^{-1} g_2 x =
      x^{-1} g_1 g_2 x = (g_1 g_2)^x$.
    \item $g^x (g^{-1})^x = (gg^{-1})^x = e^x = e \To (g^x)^{-1} = (g^{-1})^x$.
  \end{enumerate}
\end{proof}

\begin{proposition}
  Отношение сопряжённости "--- это отношение эквивалентности.
\end{proposition}
\begin{proof}~
  \begin{enumerate}
  \item Рефлексивность: $g = g^e$.
  \item Симметричность: $g^x$ сопряжён к $g$, то $g = (g^x)^{x^{-1}}$.
  \item Транзитивность: $g_2 = g_1^x$, $g_3 = g_2^y$, то $g_3 = (g_1^x)^y =
    g_1^{xy}$. \qedhere
  \end{enumerate}
\end{proof}

\begin{definition}
  Класс элемента \( g \)
  относительно этого отношения "---
  \emph{класс сопряжённости} этого
  элемента $g$. Обозначение: $g^G$.
\end{definition}

\begin{proposition}
  Пусть $H < G$.
  Тогда $H \NSG G \oTTo$ $H$ есть объединение
  нескольких классов сопряжённости.
\end{proposition}
\begin{proof}
  $H \NSG G \oTTo \Forall{g \in G} H = g^{-1} H g \To$ вместе с любым элементом
  $h \in H$, $h^G \subseteq H$ $\To H = \bigcup_{h \in H} h^G$.

  Наоборот, если $H = \bigcup_{\alpha \in A} g_\alpha^G$,
  то $g^{-1} Hg =
  \bigcup_{\alpha \in A} (g_\alpha^G)^g =
  \bigcup_{\alpha \in A} g_\alpha^G = H$.
\end{proof}

\begin{exercise}
  Пусть $g_1, g_2 \in G$. Тогда $g_1^G \cdot g_2^G$ "--- объединение нескольких
  классов сопряжённости, но не обязательно одного.
\end{exercise}
\end{document}

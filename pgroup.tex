\documentclass[main]{subfiles}

\begin{document}
% --- 2016-09-30 ---
\section{$p$-группы}
\begin{definition}
  Пусть $p$ --- простое число. Группа $G$ называется $p$-группой, если
  $|G| = p^n$, $n \in \Natural \setminus \{ 0 \}$.
\end{definition}
\begin{example}
  Если $|G| = p$, то по теореме Лагранжа
  порядок $e \ne g \in G$ равен $p$, т. е.
  $|<g>| = p = |G| \To G = \langle g \rangle$.
  Значит, $G$ циклическая и абелева.
\end{example}

\begin{theorem}
  Пусть $G$ --- $p$-группа. Тогда $Z(G) \ne \{ e \}$.
\end{theorem}
\begin{proof}
  Рассмотрим действие \( G \) на себя сопряжением.
  Орбиты "--- классы сопряжённости,
  если \( g \in Z(G) \), то \( g^G = \{ g \} \),
  иначе \( g \notin Z(G) \To
  1 < |g^G| = |G : C_G(g)| = p^k \).
  \( k \) зависит от \( g \), но \( k \ge 1 \).
  Выберем представителей \( g_i \) для каждого
  класса сопряжённости, \( i \in \{ 1, \dots, n \} \),
  тогда по формуле орбит:
  \[
    p^n = |G| = \sum_{i = 1}^n |g_i^G| =
    |Z(G)| + {\sum_{|g_i^{G}| > 1} |G : C_G(g_i)|}.
  \]
  Второе слагаемое "---
  сумма нетривиальных степеней \( p \),
  но тогда \( p \divides |Z(G)| \To |Z(G)| > 1 \).
\end{proof}

\begin{theorem}
  Пусть \( G \) "--- не абелева группа.
  Тогда \( \Factor{G}[Z(G)] \) "--- не циклическая.
\end{theorem}
\begin{remark}
  \( Z(G) \NSG G \To \) можем рассматривать факторгруппу.
  Условие неабелевости важно, так как
  иначе \( G = Z(G) \),
  и тогда \( \Factor{G}[Z(G)] \) тривиальна, а потому циклична.
\end{remark}
\begin{proof}
  Пусть это не верно.
  Положим \( Z = Z(G) \),
  и \( \Factor{G}[Z] = \langle a Z \rangle \),
  \( a \in G \To \) все левые смежные классы имеют вид
  \( a^k \cdot Z \), \( k \in \Integer \),
  т. к. \( \Factor{G}[Z] \) циклична.
  Рассмотрим произвольные \( g, h \in G \):
  \( g = a^k x \), \( h = a^l y \),
  \( k, l \in \Integer \), \( x, y \in Z \).
  Но тогда \( gh = a^k x a^l y = a^l y a^k x = hg \),
  ведь \( x \) и \( y \) коммутируют со всеми элементами.
  Значит, \( G \) абелева, противоречие.
\end{proof}

\begin{corollary}
  Если \( |G| = p^2 \), где \( p \) --- простое,
  то \( G \) "--- абелева.
\end{corollary}
\begin{proof}
  \( G \) "--- \( p \)-группа \( \To Z(G) \ne \{ e \} \),
  т. е. \( |Z(G)| = p \) или \( |Z(G)| = p^2 \).
  Во втором случае \( Z(G) = G \),
  т. е. \( G \) "--- абелева.
  Если же \( |Z(G)| = p \), то
  \( |\Factor{G}[Z(G)]| = p \To \Factor{G}[Z(G)] \) "---
  циклическая, тогда \( G \) всё равно должна быть абелевой.
\end{proof}

\begin{example}
  Рассмотрим в \( GL_3(\Integer_p) \)
  подгруппу \( UT_3(\Integer_p) \)
  унитреугольных матриц, т. к. матриц вида
  \[
    \begin{pmatrix}
      1 & a & b \\
      0 & 1 & c \\
      0 & 0 & 1
    \end{pmatrix},
  \]
  это действительно подгруппа.
  \( |UT_3(\Integer_p)| = p^3 \),
  и она не абелева, т. к.
  \begin{gather}
    \begin{pmatrix}
      1 & 1 & 0 \\
      0 & 1 & 0 \\
      0 & 0 & 1
    \end{pmatrix} 
    \cdot
    \begin{pmatrix}
      1 & 0 & 0 \\
      0 & 1 & 1 \\
      0 & 0 & 1
    \end{pmatrix} 
    =
    \begin{pmatrix}
      1 & 1 & 1 \\
      0 & 1 & 1 \\
      0 & 0 & 1
    \end{pmatrix} \\
    \begin{pmatrix}
      1 & 0 & 0 \\
      0 & 1 & 1 \\
      0 & 0 & 1
    \end{pmatrix} 
    \cdot
    \begin{pmatrix}
      1 & 1 & 0 \\
      0 & 1 & 0 \\
      0 & 0 & 1
    \end{pmatrix} 
    =
    \begin{pmatrix}
      1 & 1 & 0 \\
      0 & 1 & 1 \\
      0 & 0 & 1
    \end{pmatrix} \\
  \end{gather}
\end{example}

\begin{exercise}
  Какой у \( Z(UT_3(\Integer_p)) \) центр?
  \( |Z(UT_3(\Integer_p))| \in \{ p, p^2 \} \).
\end{exercise}

\end{document}

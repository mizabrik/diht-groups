\documentclass[main]{subfiles}

\begin{document}
\subsection{Единственность представления}
Пусть
\begin{gather}
  A = \Integer^t \times \Integer_{p_1^{\alpha_1}} \times
  \dots \times \Integer_{p_s^{\alpha_s}} , \\
  B = \Integer^u \times \Integer_{q_1^{\beta_1}} \times
  \dots \times \Integer_{q_r^{\beta_r}},
\end{gather}
\( p_i \), \( q_i \) "---
простые.
Пусть \( A \cong B \), тогда хотим доказать,
что \( t = u \), \( r = s \) и наборы
\( (p_1^{\alpha_1}, \dots, p_s^{\alpha_s}) \) и
\( (q_1^{\beta_1}, \dots, q_r^{\beta_r}) \) совпадают
с точностью до перестановки.

\begin{corollary}
  \( T(A) \cong T(B) \), \( t = u \).
\end{corollary}
\begin{proof}
  Первое утверждение очевидно. Для второго:
  \( \Integer^t \cong \Factor{A}[T(A)] \cong
  \Factor{B}[T(B)] \cong \Integer^u \).
  Итак, в этой группе существует базис из
  \( t \) и \( u \) элементов \( \To t = u \).
\end{proof}

В дальнейшем можно считать, что \( A = T(A) \),
\( B = T(B) \) (т. е. \( t = u = 0 \)).

\begin{proposition}
  Пусть \( A = \Integer_{p^{\alpha_1}} \times
  \Integer_{p^{\alpha_2}} \times
  \dots 
  \Integer_{p^{\alpha_k}} \times
  \Integer_{p_2^{\gamma_2}} \times
  \dots \times \Integer_{p_l^{\gamma_l}} \),
  где \( p_2, \dots, p_l \) отличны от \( p \).
  Тогда (единственная) силовская \( p \)-подгруппа
  в \( A \) есть \( P_p = \Integer_{p^{\alpha_1}} \times
  \dots \Integer_{p^{\alpha_k}} \times \{ 0 \}^{l-1} \).
\end{proposition}
\begin{proof}
  Ясно, что \( P_p < A \). \( |A| = p^{\alpha_1 + \dots + \alpha_k}
  \cdot \prod_{i = 2}^l p_i^{\gamma_i} \), а
  \( |P_p| = p^{\alpha_1 + \dots + \alpha_k} \).
  Значит, \( P_p \) "--- силовская \( p \)-подгруппа.
  Наконец, \( P_p \NSG A \To N_p = 1 \).
\end{proof}
\begin{corollary}
  Достаточно доказать совпадение наборов
  \( (p_i^{\alpha_i}) \) и \( (q_j^{\beta_j}) \)
  для случая \( p_1 = \dots = p_s = q_1 = \dots = q_r \).
\end{corollary}

\begin{proposition}
  Пусть \( A = \Integer_{p^{\alpha_1}} \times
  \dots \Integer_{p^{\alpha_s}} \),
  \( B = \Integer_{p^{\beta_1}} \times
  \dots \Integer_{p^{\beta_r}} \).
  Тогда, если \( A \cong B \), то
  наборы \( (\alpha_1, \dots, \alpha_s) \) и
  \( (\beta_1, \dots, \beta_r) \) совпадают
  (с точностью до перестановки).
\end{proposition}
\begin{proof}
  Индукция по \( |A| \). Если \( |A| = p \),
  то \( A \cong B \cong \Integer_p \).

  Пусть \( |A| > p \). Тогда рассмотрим
  \( pA = p\Integer_{p^{\alpha_1}} \times \dots \times
  p \Integer_{p^{\alpha_s}}
  \cong \Integer_{p^{\alpha_1 - 1}}  \times \dots
  \times \Integer_{p^{\alpha_s - 1}} \) и
  \( pB = p \Integer_{p^{\beta_1 - 1}} \times \dots \times
  p \Integer_{p^{\beta_r - 1}} \).
  Тогда, т. к. \( pA \cong pB \),
  \( |pA| = |pB| \). Но \(|pA| = \frac{|A|}{p^s} \)
  и \( |pB| = \frac{|B|}{p^r} \). Итак,
  \( s = r \).

  Кроме того, к \( pA \) и \( pB \) можно применить предположение
  индукции, получив, что наборы
  \( (\alpha_1 - 1, \dots, \alpha_s - 1) \) и
  \( (\beta_1 - 1, \dots, \beta_r - 1) \) совпадают
  с точностью до перестановки и, возможно, выкидывания
  нулей из этих наборов (случай \( \alpha_i - 1 = 0 \)
  соответствует тривиальному сомножителю \( \Integer_{p^0} \),
  который можно выкинуть).
  Но, так как \( s = r \), нулей в них одинаковое количество
  \( \To \) он совпадают с точностью до перестановки,
  а значит, совпадают и наборы
  \( (\alpha_1, \dots, \alpha_s) \) и
  \( (\beta_1, \dots, \beta_r) \).

  Если \( pA = \{ 0 \} \), то \( \alpha_1 = \dots = \alpha_s = 
  \beta_1 = \dots = \beta_r = 1 \), и утверждение
  также верно.
\end{proof}

\begin{theorem}
  Пусть \( A \) "--- конечно порождённая абелева группа.
  Тогда
  \[
    A = \Integer^t \times \Integer_{p_1^{\alpha_1}} \times
    \dots \times \Integer_{p_s^{\alpha_s}},
  \]
  где \( t \ge 0 \), \( s \ge 0 \), \( p_i \) "---
  простые числа, \( \alpha_i \ge 1 \).
  В любых таких представлениях группы \( A \)
  совпадают значения \( t \),
  а также наборы \( (p_1^{\alpha_1}, \dots, p_s^{\alpha_s}) \)
  (с точностью до перестановки).
\end{theorem}
\begin{remark}
  Таких разложений может быть много.
  \begin{enumerate}
    \item В группе \( \Integer^t \) существует много базисов,
      любой базис \( a_1, \dots, a_t \) даёт прямое
      разложение \( \Integer^t = \langle a_1 \rangle \times
      \dots \times \langle a_t \rangle \).
    \item В группе \( \Integer_p \times \Integer_p \)
      можно выбрать любые два элемента \( a \) и \( b \),
      так что \( a \notin \langle b \rangle \),
      \( b \notin \langle a \rangle \) (тогда \( a, b \ne 0 \)).
      В таком случае, \( |\langle a, b \rangle| = p^2 \), т. е.
      \( \langle a, b \rangle = \Integer_p \times \Integer_p =
      \langle a \rangle \times \langle b \rangle \).
      По сути, \( \Integer_p^2 \) "--- это двумерное пространство
      над полем \( \Integer_p \), а \( (a, b) \) "--- базис в нём.
    \item Пусть \( A = \Integer \times \Integer_2 \).
      Выберем \( a = (1, 1) \) и \( b = (0, 1) \).
      Тогда \( \langle a \rangle \cong \Integer \),
      \( \langle b \rangle \cong \Integer_2 \),
      и \( A = \langle a \rangle \times \langle b \rangle \).
      ("<канонический"> выбор "--- это \( (1, 0) \), \( (0, 1) \)).
  \end{enumerate}
\end{remark}

\begin{exercise}
  Мы доказали, что любая конечно порождённая группа \( A \)
  есть \( A \cong \Integer^t \times \Integer_{m_1} \times
  \dots \times \Integer_{m_k} \), где \( t \ge 0 \),
  \( k \ge 0 \), \( m_i > 1 \) и \( m_1 \divides m_2 \divides
  \dots \divides m_k \).
  Докажите, что и в этом представлении все параметры
  восстанавливаются однозначно.
\end{exercise}
\begin{remark}
  Ещё одно следствие из теоремы о существовании согласованных базисов.
  Пусть \( A = \Integer^k \), \( B \) "--- подгруппа в \( A \),
  порождённая столбцами  некоторой матрицы \( M = (m_{ij}) \in
  \Matrices[\Integer]{k}{n} \).
  Тогда в \( A \) и \( B \) существуют согласованные базисы
  \( (a_1, \dots, a_k) \) и \( (b_1, \dots, b_l) \) такие, что
  \( b_i = d_i a_i \), \( d_1 \divides d_2 \divides \dots \divides d_l \).

  Замена базиса в \( A \) соответствует умножению \( M \) на матрицу
  перехода \( S \in \Matrices[\Integer]{k}{k} \), \( \det S = \pm 1 \).
  Можно показать, что переход к \( (b_1, \dots, b_l, 0, \dots, 0) \)
  также можно осуществить с помощью матрицы перехода
  \( T \in \Matrices[\Integer]{n}{n} \), \( \det T = \pm 1 \).
  таким образом,
  \[
    S M T = Diag(d_1, d_2, \dots, d_l, 0, \dots).
  \]
  Полученная матрица называется \emph{смитовой нормальной формой}
  матрицы \( M \).
  В этом случае \( \Factor{A}[B] = \Integer_{d_1} \times \dots
  \times \Integer_{d_l} \times \Integer^{k - l} \),
  где \( d_1, \dots, d_l \) определяются однозначно.
  Таким образом, смитова нормальная форма матрицы \( M \)
  единственна.
\end{remark}

\begin{exercise}
  Восполните пробелы.
\end{exercise}
\begin{exercise}
  \( d_i = \GCD(\text{миноры матрицы \( M \) порядка \( i \)}) \).
\end{exercise}
\end{document}

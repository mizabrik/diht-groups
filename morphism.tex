\documentclass[main]{subfiles}

\begin{document}
\section{Гомоморфизмы групп}

\begin{definition}
  Пусть $G$, $H$ "--- группы. Отображение $\phi : G \to H$ называется
  \emph{гомоморфизмом групп}, если $\Forall{g_1, g_2 \in G}
  \phi(g_1 g_2) = \phi(g_1) \phi(g_2)$.
  \emph{Образ} гомоморфизма "--- это
  $\Img \phi = \phi(G) = \{ \phi(g) \mid g \in G \}$.
  \emph{Ядро} гомоморфизма $\Ker \phi = \phi^{-1}(e_H)$.
  Гомоморфизм называется \emph{эпиморфизмом}, если $\Img \phi = H$, и
  \emph{мономорфизмом}, если $\phi$ "--- инъекция.
\end{definition}

\begin{proposition}
  Пусть $\phi : G \to H$ "--- гомоморфизм. Тогда:
  \begin{enumerate}
    \item $\phi(e_G) = e_H$.
    \item $\phi(g^{-1}) = \phi(g)^{-1}$.
  \end{enumerate}
\end{proposition}
\begin{proof}~
  \begin{enumerate}
    \item $\phi(e) = \phi(e^2) = \phi(e) \phi(e) \To e = \phi(e)$.
    \item $\phi(g^{-1})\phi(g) = \phi(g^{-1}g) = \phi(e) = e$.
  \end{enumerate}
\end{proof}

\begin{proposition}
  Гомоморфизм $\phi : G \to H$ является мономорфизмом $\oTTo \Ker \phi = \{e\}$.
\end{proposition}
\begin{proof}
  $\phi(e) = e \To e \in \Ker \phi$. Если $\phi$ "--- мономорфизм, то
  $\Forall{e \ne g \in G} \phi(g) \ne \phi(e) \To \Ker \phi = \{e\}$.
  Если $\Exists{g_1 \ne  g_2} \phi(g_1) = \phi(g_2) \To \phi(g_1^{-1} g_2)
  = \phi(g_1)^{-1} \phi(g_2) = e \To e \ne g_1^{-1} g_2 \in \Ker \phi$.
\end{proof}

\begin{examples}~
  \begin{enumerate}
    \item $\phi : G \to H$, $\phi(g) = e$.
    \item $\phi : \Integer \to \Integer_n$, $\phi(a) = a \pmod{n}$;
      $\Ker \phi = n \Integer$.
    \item $\phi : GL_n(F) \to F^*$, $\phi(A) = \det A$; $\Ker \phi = SL_n$.
    \item Изоморфизм является гомоморфизмом. В частности, существуют изоморфизмы
      группы \emph{на себя} (\emph{автоморфизмы}). Например, если $x \in G$, то
      $\phi_x : g \mapsto g^x$ "--- автоморфизм.
  \end{enumerate}
\end{examples}

\begin{proposition}
  Пусть $\phi : G \to H$ "--- гомоморфизм групп. Тогда $\Img \phi < H$,
  $\Ker \phi \NSG G$.
\end{proposition}
\begin{proof}
  Если $h_1, h_2 \in \Img \phi$, то $h_i = \phi(g_i)$. $g_i \in G \To
  h_1 h_2 = \phi(g_1 g_2) \in \Img \phi$, $h_1^{-1} = \phi(g_1^{-1}) \in
  \Img \phi$. Значит, $\Img \phi < H$.

  Если $g_1, g_2 \in \Ker \phi$, то $\phi(g_1 g_2) = \phi(g_1)\phi(g_2) = e$,
  $\phi(g_1^{-1}) = e \To g_1 g_2, g_1^{-1} \in \Ker \phi \To \Ker \phi < G$.
  Кроме того, $\phi(x^{-1} \cdot \Ker \phi \cdot x) = \phi(x)^{-1} \cdot
  \phi(\Ker \phi) \cdot \phi(x) = \phi(x)^{-1} \phi(x) = e \To
  \Forall{x \in G} x^{-1} \Ker \phi \cdot x \subseteq \Ker \phi \To
  \Ker \phi \NSG G$.
\end{proof}

\begin{remark}
  Если $K < H$, то $\phi : K \to H$, $\phi(k) = k$ "--- это гомоморфизм,
  $\Img \phi = K$.
\end{remark}

Пусть $K \NSG G$. Рассмотрим $\Factor{G}[K]$ "---
множество левых смежных классов по $K$.
Если $aK, bK \in \Factor{G}[K]$,
то $aK \cdot bK = a (Kb) K =  abKK = abK \in \Factor{G}[K]$
(в силу нормальности).

\begin{theorem}
  $(\Factor{G}[K], \cdot)$ "--- группа.
\end{theorem}
\begin{proof}
  Ассоциативность следует из ассоциативности в $G$.
  Нейтральный элемент "--- это $K = eK$,
  обратный к $aK$ --- $a^{-1} K$.
\end{proof}

\begin{definition}
  Полученная группа "--- \emph{факторгруппа}
  группы $G$ по нормальной подгруппе $K$.
\end{definition}

\begin{theorem}
  Отображение $\pi : G \to \Factor{G}[K]$, $\pi(g) = gK$,
  является эпиморфизмом,
  $\Ker \pi = K$.
\end{theorem}
\begin{proof}
  $\pi(g_1 g_2) = g_1 g_2 K =
  g_1 K \cdot g_2 K = \pi(g_1) \pi(g_2) \To$
  $\pi$ "--- гомоморфизм.
  Любой $gK \in \Factor{G}[K]$ есть $\pi(g) \To \pi$ --- эпиморфизм.
  $g \in \Ker \pi \oTTo \pi(g) = gK
  = K \oTTo g \in K$. Итак, $\Ker \pi = K$.
\end{proof}

\begin{definition}
  $\pi$ "--- \emph{естественный} эпиморфизм $G \to \Factor{G}[K]$.
\end{definition}

\begin{theorem}[основная теорема о гомоморфизмах групп]
  Пусть $\phi : G \to H$ "--- гомоморфизм групп, $\Ker \phi = K$.
  Тогда $K \NSG G$ и $\Img \phi \cong \Factor{G}[K]$. \\
  Наоборот, если $K \NSG G$,
  то существует эпиморфизм групп $\pi : G \to \Factor{G}[K]$,
  $\Ker \phi = K$.
\end{theorem}
\begin{proof}
  Осталось доказать изоморфность образу: $\Img \phi \cong \Factor{G}[K]$.

  Определим $\psi: \Factor{G}[K] \to \Img \phi$ так:
  $\psi(aK) = \phi(aK) =
  \phi(a) \phi(K) = \phi(a)$.
  Если $\phi(a) \in \Img \phi$,
  то $\phi(a) = \psi(aK) \To \psi$ "--- сюрьекция.
  Если $\psi(aK) = \psi(bK)$, то
  $\phi(a) = \phi(b) \To \phi(a^{-1} b) = e \To
  a^{-1} b \in \Ker \phi = K
  \To b \in aK \To bK = aK$.
  Итак, $\psi$ "--- инъекция.

  Теперь покажем
  что $\psi$ сохраняет операции:
  $\psi(aK) \cdot \psi(bK) =
  \phi(a) \phi(b) = \phi(ab) = \psi(abK) = \psi(aK \cdot bK)$.
  Значит, $\psi$ "--- изоморфизм.
\end{proof}

\begin{remark}
  Если $h \in \Img \phi$, то $\psi^{-1}(h) = \phi^{-1}(h)$.
\end{remark}

\begin{examples}~
  \begin{enumerate}
    \item $\phi : \Integer \to \Integer_n$, $\phi(a) = a \mod{n}$. $\Ker\phi = n\Integer$,
      $\Img \phi = \Integer_n \To \Integer_n \cong \Integer/n\Integer$.
    \item $\det: GL_n(F) \to F^*$. $\Img \det = F^*$; $\Ker \det = SL_n(F) \To
      GL_n(F)/SL_n(F) \cong F^*$.
  \end{enumerate}
\end{examples}

\begin{exercise}
  $S_n/A_n \cong$ ?.
\end{exercise}

\begin{theorem}[первая теорема об изоморфизме]
  Пусть $H \NSG G$, $K < G$. Тогда:
  \begin{enumerate}
    \item $HK = KH < G$.
    \item $K \cap H \NSG K$.
    \item $\Factor{HK}[H] \cong \Factor{K}[(H\cap K)]$.
  \end{enumerate}
\end{theorem}
\begin{proof}
  Первый пункт уже доказан. Второй и третий следуют из рассмотрения естественного
  эпиморфизма: $\pi : G \to \Factor{G}[H]$.
  Тогда $\pi_{HK} \coloneq \pi \bigr|_{HK}$ и
  $\pi_K \coloneq \pi \bigr|_K$ "--- гомоморфизмы групп.
  $\Img \pi_{HK} = \pi(HK) = \pi(H) \pi(K) = \pi(K) = \Img \pi_K$.
  $\Ker \pi_{HK} = H$, $\Ker \pi_K = K \cap H$.
  Тогда $\Factor{HK}[H] \cong \Img \pi_{HK} =
  \Img \pi_{K} \cong \Factor{K}[(H \cap K)]$.
\end{proof}

\begin{remark}
  Явный вид изоморфизма:
  $k(H \cap K) \in \Factor{K}[(H \cap K)]
  \otto kH \in \Factor{HK}[H]$.
\end{remark}

% --- 2015-09-15 ---
\begin{remark}
  $K \cap H \NSG K$, поскольку $K \cap H = \Ker \pi_K$.
\end{remark}

\begin{example}
  Пусть $G = S_4$,
  $H = V_4 = \{ \id, \Cycle{1&2}\Cycle{3&4},
  \Cycle{1&3}\Cycle{2&4}, \Cycle{1&4}\Cycle{2&3} \}$.
  Нетрудно поверить, что $V_4 \NSG S_4$
  ($V_4$ называется четверной группой Клейна).
  Положим $K = S_3 < S_4$.
  $H \cap K = \{ id \}$.
  Это значит, что
  $\Forall{h _i \in H, k_i \in K}
  h_1 k_1 = h_2 k_2
  \To H \ni h_2^{-1} h_1 = k_2 k_1^{-1} \in K
  \To h_1 = h_2, k_1 = k_2$.
  Значит, $|HK| = |H| \cdot |K| = 24 \To HK = S_4$.
  Применяя первую теорему об изоморфизме,
  имеем $\Factor{S_4}[V_4] = H\Factor{K}[H]
  \cong \Factor{K}[H\cap K] = \Factor{S_3}[\{id\}] \cong S_3$.
\end{example}

\begin{theorem}[Вторая теорема об изоморфизме, или теорема о соответствии]
  Пусть $G$ "--- группа, $H \NSG G$,
  обозначим \( \overline{G} \coloneq \Factor{G}[H] \).
  Тогда:
  \begin{enumerate}
    \item Для подгруппы \( K < G \) такой,
      что \( H < K \), обозначим
      \( \overline{K} \coloneq \Factor{K}[H] \).
      Тогда \( \overline{K} < \overline{G} \).
    \item Соответствие $K \otto \overline{K}$ "---
      биекция между
      подгруппами в $G$, содержащими $H$,
      и подгруппами в $\bar{G}$.
    \item Если \( H < K < G \), то
      $K \NSG G \oTTo \overline{K} \NSG \overline{G}$,
      и в этом случае
      $\Factor{G}[K] \cong \Factor{\overline{G}}[\overline{K}]$.
  \end{enumerate}
\end{theorem}
\begin{proof}
  Рассмотрим естественный эпиморфизм
  $\pi : G \to \Factor{G}[H]$ ($\pi(g) = gH$).
  Тогда $\pi(K) = K\Factor{H}[H] =
  \Factor{K}[H] = \overline{K}$.
  Наоборот, если $L < \overline{G}$,
  то $H < \pi^{-1}(L) < G$
  (если $a, b \in \pi^{-1}(L)$,
  то $ab, a^{-1} \in \pi^{-1}(L)$).
  При этом, $\pi(\pi^{-1}(L)) = L$,
  ибо $\pi$ "--- сюрьекция;
  кроме того,
  для любой такой \( K < G \)
  $$\pi^{-1}(\pi(K)) = \pi^{-1}(\overline{K}) =
  \bigcup_{kH \in \overline{K}} kH =
  \bigcup_{k \in K} kH = K.$$
  Итак, $\pi$ осуществляет требуемую биекцию
  $K \to \overline{K}$.

  Если $K \NSG G$, то
  $g^{-1}Kg = K \To \pi(g)^{-1}\pi(K) \pi(g) = \pi(K)$
  для любого $g \in G$.
  Поскольку $\pi$ "--- сюрьекция,
  $\pi(K) = \overline{K} \NSG \overline{G}$.

  Пусть $\overline{K} \NSG \overline{G}$.
  Тогда существует естественный эпиморфизм
  $\pi' : \overline{G} \to \overline{G}/\overline{K}$.
  Рассмотрим $\pi' \circ \pi:
  G \to \overline{G}/\overline{K}$.
  Это "--- эпиморфизм,
  при этом $\Ker(\pi' \circ \pi) =
  \pi^{-1}(\pi'^{-1}(e)) = \pi^{-1}(\overline{K}) = K$.
  Значит,
  $\overline{G}/\overline{K} =
  \Img(\pi' \circ \pi) \cong G/\Ker(\pi' \circ \pi) =
  \Factor{G}[K]$
  по основной теореме о гомоморфизмах
  (и $K = \Ker(\pi' \circ \pi) \NSG G$).
\end{proof}

\begin{example}
  Пусть $m, n \in \Natural$.
  Тогда $\Integer \NGS n \Integer \NGS mn \Integer$
  (и $\Integer \NGS mn \Integer$).
  Значит, $\Factor{\Integer}[n \Integer] = \Integer_n$,
  $\Factor{\Integer}[mn \Integer] = \Integer_{mn}$,
  а $n (\Factor{\Integer}[mn\Integer]) = n \Integer_{mn}$.
  Следовательно, $\Factor{\Integer_{mn}}[n \Integer_{mn}]
  = (\Integer/mn\Integer)/
  (n\Integer/mn\Integer) \cong \Integer/n\Integer = \Integer_n$.
  Как применить пункты 1 и 2?
  Все подгруппы $\Integer$ имеют вид $k \Integer$,
  $k \in \Integer$.
  Тогда все подгруппы $\Integer_n = \Factor{\Integer}[n \Integer]$
  есть подгруппы вида $\Factor{k \Integer}[n \Integer]$,
  где $n \Integer < k \Integer$
  (т. е. $k \divides n$).
  Значит, подгруппы в $\Integer_n$ "---
  это подгруппы вида $(n/k) \Integer_n$,
  где $k \divides n$,
  т. е. $l \Integer_n$, где $l \divides n$.
\end{example}

\end{document}

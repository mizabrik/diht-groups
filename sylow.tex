\documentclass[main]{subfiles}

\begin{document}
\section{Теоремы Силова}

Пусть $|G| = n$, $k \divides n$. В таком случае
\emph{необязательно} существует $H < G$,
$|H| = k$.
Например, в группе $A_4$ нет
$H < A_4$: $|H| = 6$.
Иначе бы $|A_4 : H| = 2
\To H \NSG A_4$,
но классы сопряженности в $A_4$
имеют порядки $1$, \( 3 \), \( 4 \), \( 4 \),
и из этих порядков
не составить $6$.

\begin{definition}
  Пусть $G$ "--- конечная группа, $|G| = n = p^k s$,
  где $p$ "--- простое, $k \ge 1$, $p \ndivides s$.
  Тогда \emph{силовской $p$-подгруппой} в $G$
  называется подгруппа $H < G$ такая,
  что $|H| = p^k$.
\end{definition}

\begin{theorem}[1-я теорема Силова]
  В любой конечной группе $G$, $p \divides n = |G|$,
  существует силовская $p$-подгруппа.
\end{theorem}
\begin{theorem}[2-я теорема Силова]
  Любая $p$-подгруппа группы $G$ содержится в
  некоторой силовской $p$-подгруппе.
  Более того, все силовские $p$-подгруппы в $G$
  сопряжены.
\end{theorem}
\begin{theorem}[3-я теорема Силова]
  Пусть $N_p$ "--- количество словских $p$-подгрупп
  в $G$. Тогда $N_p \equiv 1 \mod{p}$.
\end{theorem}

\begin{remark}
  Во-первых, из второй теоремы следует,
  что все силовские $p$-подгруппы в $G$ изоморфны.
  Во-вторых, уже известно,
  что в группе порядка $p^k$ есть подгруппы любого
  порядка $p^l$, $l \le k$. Значит, и в $G$ есть
  такие (но они не обязательно изоморфны).
\end{remark}

\begin{proof}[Доказательство 2-й теоремы при условии 1-й]
  Пусть $P$ "--- силовская $p$-подгруппа в $G$,
  $H$ "--- некоторая $p$-подгруппа в $G$, $|H| = p^t$.
  Рассмотрим действие группы $H$ на $\Omega = \Factor{G}[P]$
  левыми сдвигами: $h(gP) = hgP$.
  Пусть $\Omega = \Omega_1 \sqcup \Omega_2 \sqcup \dots \sqcup \Omega_m$ "---
  разбиение на орбиты. Тогда для любого $i$ выбрав $\omega_i \in \Omega_i$
  $|\Omega_i| = |H : \St(\omega_i)| = p^{\alpha_i}$, $\alpha_i \in \Integer_+$.
  Значит, по формуле орбит
  $s = \frac{n}{p^k} = |\Omega| = p^{\alpha_1} + \dots + p^{\alpha_m}$.
  Поскольку $p \ndivides s$, существует $\alpha_i = 0$,
  т. е. $\Omega_i = \{ gP \}$, $g \in G$.
  Но $P_1 = gP g^{-1} < G$ "--- подгруппа, сопряженная с $P$.
  Итак, $H \cdot P_1 = P_1 \To H \subseteq P_1 \To H < P_1$, $|P_1| = p^k$.
  Наконец, если $H$ "--- силовская $p$-подгруппа, то $H < g P g^{-1}$,
  $|H| = |gPg^{-1}| \To H = gP g^{-1}$, т. е. $H$ и $P$ сопряжены.
\end{proof}

\begin{proof}[Доказательство 3-й (и 1-й) теоремы]
  Пусть $\Omega = \{ M \subseteq G \mid |M| = p^k \}$,
  $G$ действует на $\Omega$ левыми сдвигами:
  $g(M) = gM \in \Omega$.
  Пусть $M \in \Omega$, $H = \St(M)$.
  Это значит, что для любого $h \in H$
  $hM = M \To H M = M$.
  Но $HM = \bigcup_{m \in M} Hm$, т. е.
  $M$ есть объединение правых смежных классов по $H$,
  откуда $|H| \divides |M| = p^k \To |H| = p^t$, $t \in \Integer$,
  а тогда $|G(M)| = |G : H| = s p^{k - t}$.

  Заметим: если $H$ "--- силовская $p$-подгруппа, то $M = HM = Hm$,
  $m \in M$. Наоборот, если $K$ "--- произвольная силовская подгруппа, то
  для множества $M = Kg$ имеем $KM = M \To K < \St(M) \To K = \St(M)$.
  Итого: любая силовская $p$-подгруппа $K < G$ является стабилизатором ровно для
  $|G : K| = s$ подмножеств "--- своих правых смежных классов.
  В то же время, любой правый смежный класс
  силовской \( p \)-подгруппы
  будет левым для некоторой (возможно, другой)
  силовской \( p \)-подгруппы,
  а его орбита "--- всеми её левыми смежными классами.

  Применим формулу орбит:
  если $\Omega = \Omega_1 \sqcup \dots \sqcup \Omega_m$ "---
  разбиение \( \Omega \) на орбиты, то
  \[
    C_n^{p^k} = |\Omega| = \sum_{i = 1}^m |\Omega_i| =
    \sum_{i = 1}^m s \cdot p^{k - t_i},
  \]
  где среди чисел $t_i$ есть
  ровно $N_p$ чисел, равных $k$;
  для остальных же
  слагаемые будут кратны \( p \).
  Значит,
  $$C_n^{p^k} \equiv N_p \cdot s \pmod{p},$$
  т. е.  $N_p \mod p$ зависит только лишь от $n$,
  а не от того,
  какую группу порядка \( n \) мы выбрали.

  Например, если $G = \Integer_n$,
  то в ней ровно одна подгруппа порядка $p^k
  \To N_p \equiv 1 \pmod{p}$ для любой $G$,
  $|G| = n$.
\end{proof}

\begin{exercise}
  Пусть $0 \le l \le k$,
  и пусть $N_p(l)$ "--- количество
  подгрупп порядка $p^l$ в группе $G$.
  Тогда $N_p(l) \equiv 1 \pmod{p}$.
\end{exercise}

\begin{proposition}
  Пусть $p < q$ "--- простые числа. Тогда любая группа
  $G$, $|G| = pq$, разрешима.
\end{proposition}
\begin{proof}
  Пусть $Q$ "--- силовская $q$-подгруппа в $G$, $|Q| = q$.
  Все силовские $q$-подгруппы сопряжены с ней,
  и их количество есть $N_q \equiv 1 \mod{q}$.
  Если $N_q = 1$, то $g^{-1} Q g = Q$ для любого $g \in G$
  $\To Q \NSG G$, $|\Factor{G}[Q]| = p$, т. е. $Q$ и $\Factor{G}[Q]$ "--- циклические
  $\To$ $G$ "--- разрешима. Иначе в любой силовской $q$-подгруппе
  найдётся $q - 1$ элементов порядка $q$, и все они различны
  $\To |G| \ge N_q \cdot (q - 1) \ge (q + 1)(q - 1) > pq$ "---
  противоречие.
\end{proof}

% --- 2016-11-17

\begin{theorem}
  Пусть \( G \) "--- конечная группа,
  \( p_1, \dots, p_k \) "--- все
  различные простые делители \(n = |G| \),
  а \( P_1, \dots, P_k \) "---
  соответствующие силовские подгруппы в \( G \).
  Тогда:
  \begin{enumerate}
    \item \( P_i \NSG G \oTTo N_{p_i} = 1 \)
    \item \( G = P_1 \times \dots \times P_k \oTTo
      \Forall{i} P_i \NSG G \)
  \end{enumerate}
\end{theorem}
\begin{proof}~
  \begin{enumerate}
    \item Если \( N_{p_i} = 1 \), то \( P_i \) "---
      единственная силовская \( p_i \)-подгруппа,
      а тогда
      \( \Forall{g \in G} g P_i g^{-1} = P_i \To P_i \NSG G \).
      Наоборот, если \( P_i \NSG G \),
      то по второй теореме Силова
      любая силовская \( p_i \)-подгруппа
      сопряжена с \( P_i \),
      т. к. совпадает с \( P_i \).
      Значит, \( N_{p_i} \) = 1.

    \item Если \( G = P_1 \times \dots \times P_k \),
      то \( P_i \NSG G \).
      Наоборот,
      если \( P_i \NSG G \),
      докажем индукцией по \( t \),
      что \( P_1 \cdot \dots \cdot P_t = P_1 \times \dots \times P_t \).
      При \( t = 1 \) "--- доказывать нечего.
      Пусть \( P_1 \cdot \dots \cdot P_{t - 1} =
      P_1 \times \dots \times P_{t - 1} \),
      тогда \( |P_1 \dots P_{t - 1}| \) делится
      только на \( p_1, \dots, p_{t - 1} \), а \( |P_t| \) делится лишь
      на \( p_t \).
      Отсюда \( |P_1 \dots P_{t - 1} \cap P_t| \)
      делит \( \GCD( |P_1 \dots P_{t - 1}|, |P_t|) \To
      P_1 \dots P_{t - 1} \cap P_t = \{ e \} \).
      Итак, \( P_1 \dots P_{t - 1}, P_t \NSG P_1 \dots P_t \),
      \( P_1 \dots P_{t - 1} \cap P_t = \{ e \} \),
      \( P_1 \dots P_{t - 1} \cdot P_t = P_1 \dots P_t \To
      P_1 \dots P_t = (P_1 \dots P_{t - 1}) \times P_t =
      P_1 \times \dots \times P_t \).
      \qedhere
  \end{enumerate}
\end{proof}
\begin{corollary}
  Любая конечная абелева группа "---
  прямое произведение своих силовских подгрупп.
\end{corollary}

\end{document}

\documentclass[main]{subfiles}

\begin{document}

\section{Коммутант}
\begin{definition}
  Пусть \( G \) "--- группа, \( x, y \in G \).
  \emph{Коммутатором} этих элементов
  называется \( [x, y] = xyx^{-1}y^{-1} \).
\end{definition}

\begin{proposition}
  Для любых \( x, y \in G \) верно:
  \begin{enumerate}
    \item \( xy = yx \oTTo [x, y] = e \)
    \item \( xy = [x, y] yx \)
    \item \( [x, y]^{-1} = [y, x] \)
    \item \( [x, y]^g = [x^g, y^g] \)
  \end{enumerate}
\end{proposition}
\begin{proof}~
  \begin{enumerate}
    \item Следует из следующего свойства.
    \item \( [x, y] \cdot yx = xy x^{-1} y^{-1} yx = xy \).
    \item \( [x, y]^{-1} = y x y^{-1} x^{-1} = [y, x] \).
    \item Следует из того, что сопряжение "--- автоморфизм.
  \end{enumerate}
\end{proof}

\begin{remark}
  Если \( \phi : G \to A \) "--- гомоморфизм
  в абелеву группу \( A \), то
  \( \phi([x, y]) = [\phi(x), \phi(y)] = e \).
\end{remark}

\begin{definition}
  Пусть \( G \) "--- группа, тогда
  \( G' = \langle \{ [x, y] \mid x, y \in G \} \rangle \)
  называется \emph{коммутантом} группы \( G \).
\end{definition}

Более общо: если \( K, H < G \), то
их взаимным коммутантом называется
подгруппа \( [K, H] = \langle
\{ [k, h] \mid k \in K, h \in H \} \rangle \).
Таким образом, \( G' = [G, G] \).

\begin{remark}
  \( \{ [x, y] \mid x, y \in G \} \) не обязательно
  является подгруппой в \( G \).
\end{remark}
\begin{exercise}
  Привести соответствующий пример.
\end{exercise}

\begin{proposition}
  Пусть \( \phi : G \to H \) "--- гомоморфизм групп.
  Тогда \( \phi(G') < H' \).
  Более того, если \( \phi \) "--- эпиморфизм,
  то \( \phi(G') = H' \).
\end{proposition}
\begin{proof}
  Для любых \( x, y \)
  \( \phi([x, y]) = [\phi(x), \phi(y)] \in H' \).
  Значит, \( \phi(\{ [x,y] \mid x, y \in G \}) \subseteq H' \To
  \phi(G') = \langle \{ \phi([x, y]) \mid x, y \in G \} \rangle
  \subseteq H' \).
  Если же \( \phi \) "--- эпиморфизм, то
  \( \Forall{a, b \in H} \Exists{x, y \in G}
  \phi(x) = a, \phi(y) = b \).
  Тогда \( [a, b] = \phi([x, y]) \in \phi(G') \To
  H' \subseteq \phi(G') \To \phi(G') = H' \).
\end{proof}

\begin{corollary}
  \( K \NSG G \To K' \NSG G \).
\end{corollary}
\begin{proof}
  Пусть \( g \in G \), \( I_g \in \Aut G \),
  \( I_g(x) = x^{g^{-1}} \).
  Тогда, т. к. \( K \NSG G \),
  то \( I_g(K) = K \), т. е.
  \( I_g \bigr|_K : K \to K \) "---
  автоморфизм группы \( K \).
  Значит, \( I_g(K') = K' \), т. е.
  \( g K' g^{-1} = K' \To K' \NSG G \).
\end{proof}

\begin{definition}
  \( G'' = (G')' \), по индукции,
  \( G^{(n)} = (G^{(n-1)})' \).
  Подгруппа \( G^{(n)} \) называется
  \emph{\(n\)-м коммутантом} группы \( G \).
\end{definition}

\begin{corollary}
  \( G' \NSG G \); более того, \( G^{(n)} \NSG G \).
\end{corollary}
\begin{proof}
  Индукция по \( n \). При \( n = 0 \),
  \( G \NSG G \). Шаг индукции "---
  предыдущее следствие.
\end{proof}

\begin{theorem}
  Для группы \( G \) верно:
  \begin{enumerate}
    \item \( \Factor{G}[G'] \) "--- абелева группа.
    \item Если \( G' < K < G \), то \( K \NSG G \).
    \item Если \( K \NSG G \), причём \( \Factor{G}[K] \) "---
      абелева, то \( G' < K \).
  \end{enumerate}
\end{theorem}
\begin{remark}
  Это значит, что \( G' \) "--- наименьшая  по
  включению нормальная подгруппа,
  факторгруппа по которой Абелева.
\end{remark}
\begin{proof}~
  \begin{enumerate}
    \item Пусть \( \pi : G \to \Factor{G}[G'] \) "---
      канонический эпиморфизм, тогда
      \( \Forall{x, y \in G} [\pi(x), \pi(y)] =
      \pi([x, y]) = e \To \)
      \( \pi(x) \) и \( \pi(y) \) коммутируют.
      Поскольку \( \pi \) "--- эпиморфизм,
      то \( \Factor{G}[G'] \) "--- абелева.
    \item Рассмотрим \( \pi(K) = \Factor{K}[G'] <
      \Factor{G}[G'] \). Поскольку \( \Factor{G}[G'] \) "---
      абелева, то \( \Factor{K}[G'] \NSG \Factor{G}[G'] \), и
      по второй теореме об изоморфизме \( K \NSG G \).
    \item Пусть \( \Factor{G}[K] \) "--- абелева.
      Рассмотрим канонический эпиморфизм
      \( \pi' : G \to \Factor{G}[K] \).
      Тогда \( \pi'([x, y]) = [\pi'(x), \pi'(y)] = e \To
      [x, y] \in \Ker \pi' = K \) для произвольных
      \( x, y \in G \). Значит, \( G' < K \).
  \end{enumerate}
\end{proof}

\begin{remark}
  Наоборот, если \( G' < K < G \),то
  \( \Factor{G}[K] \cong
  \Factor{(\Factor{G}[G'])}[(\Factor{K}[G'])] \) "---
  абелева.
\end{remark}
\begin{remark}
  В конце доказательства мы, по сути, увидели,
  что для любого гомоморфизма \( \phi : G \to A \),
  где \( A \) "--- абелева, \( \Ker \phi > G' \).
\end{remark}

% 2016-10-13
\begin{exercise}
  Пусть $H \NSG G$, $K = [G, H]$. Тогда $K$ "--- наименьшая
  подгруппа такая, что $\Factor{H}[K] < Z(\Factor{G}[K])$.
\end{exercise}

\begin{definition}
  Пусть $G$ "--- группа, $M \subseteq G$.
  Тогда \emph{нормальная подгруппа, порождённая множеством $M$},
  есть
  \[
    \GenNGroup{M} =
    \bigcap_{H \NSG G, M \subseteq H} H.
  \]
\end{definition}

\begin{proposition}
  $\GenNGroup{M} = \GenGroup{M^G}$,
  где $M^G = \{ m^g \mid m \in M, g \in G \}$.
\end{proposition}
\begin{proof}
  Если $H \NSG G$, $M \subseteq H$, то
  $M^G \subseteq H$.
  Значит,
  $\GenGroup{M^G} =
  \bigcap_{H < G, H \supseteq M^G} H \subseteq
  \bigcap_{H \NSG G,  H\supseteq M} H =
  \GenNGroup{M}$.
  Наоборот, $\GenGroup{M^G} \NSG G$,
  т. к. $\Forall{g \in G}
  \langle M^G \rangle^g = \langle M^{Gg} \rangle =
  \langle M^G \rangle$,
  поэтому $\GenNGroup{M} \subseteq \GenGroup{M^G}$.
\end{proof}

\begin{proposition}
  Пусть $G = \GenGroup{M}$.
  Тогда $G' = \GenNGroup{\{ [m_1, m_2] \mid
  m_1, m_2 \in M \}}$.
\end{proposition}
\begin{proof}
  Обозначим правую часть равенства через $H$.
  Раз $G' \NSG G$
  и $[m_1,m_2] \in G'$
  для любых \( m_1, m_2 \in M \),
  получаем $H < G'$.

  Наоборот, рассмотрим $G/H$
  и канонический эпиморфизм $\pi: G \to G/H$;
  $[\pi(m_1), \pi(m_2)] = \pi([m_1,m_2]) = e$
  для произвольных \( m_1, m_2 \in M \).
  Итак, $G/H = \langle \pi(M) \rangle$,
  и любые два элемента из $\pi(M)$ коммутируют.
  Значит, $G/H$ "--- абелева,
  откуда $G' < H$.

  Значит, $G' = H$.
\end{proof}

\begin{exercise}
  Приведите пример,
  когда $G' \ne \GenGroup{\{ [m_1, m_2] \mid m_1, m_2 \in M \}}$.
\end{exercise}

\begin{remark}
  Для группы $G$ обе подгруппы $Z(G)$ и $G'$ показывают, насколько "<далека"> $G$
  от абелевой.
\end{remark}

\end{document}

\documentclass[main]{subfiles}

\begin{document}
\section{Действие группы на множество}
\begin{definition}\label{def:action-1}
  Пусть $G$ "--- группа, $\Omega$ "--- множество. Говорим,
  что определено \emph{действие группы} $G$ на множестве $\Omega$,
  если определено отображение $G \times \Omega \to \Omega$
  (т. е. для любых $g \in G$, $\omega \in \Omega$ определён $g(\omega)
  \in \Omega$), удовлетворяющее следующим свойствам:
  \begin{enumerate}
    \item $(gh)(\omega) = g(h(\omega))$
    \item $e(\omega) = \omega$
  \end{enumerate}
\end{definition}

\begin{definition}\label{def:action-2}
  \emph{Действие группы} $G$ на множество $\Omega$ "--- это гомоморфизм
  $G \to S(\Omega)$, где $S(\Omega)$ "--- группа всех биекций множества
  $\Omega$ в себя.
\end{definition}

\begin{proposition}
  Данные определения эквивалентны.
\end{proposition}
\begin{proof}
  Пусть задано действие $G$ на $\Omega$
  по определению \ref{def:action-1}.
  Положим
  для $g \in G$
  $I_g : \Omega \to \Omega$,
  $I_g(\omega) = g(\omega)$.
  Тогда  $I_{gh}(\omega) = (gh)(\omega) = g(h(\omega)) = I_g \circ I_h(\omega)$,
  т. е. $I_{gh} = I_g \circ I_h$.
  Кроме того, $I_g \circ I_{g^{-1}} = I_e =
  I_{g^{-1}} I_g$,
  причём $I_e = \id$.
  Т. е. $I_{g^{-1}}$ "--- это обратное отображение к $I_g$,
  а значит $I_g$ "--- биекция.
  Итак, $g \mapsto I_g$ "---
  это требуемый гомоморфизм.

  Наоборот,
  пусть $g \mapsto I_g$ "---
  гомоморфизм $G \to S(\Omega)$,
  то положим $g(\omega) = I_g(\omega)$.
  Тогда $(gh)(\omega) = I_{gh}(\omega) = I_g \circ I_h(\omega) = g(h(\omega))$,
  т. е. первое свойство доказано.
  Кроме того, $I_e = id$,
  т. е. $e(\omega) = \omega$.
\end{proof}

\begin{definition}
  Пусть $\phi : G \to S(\Omega)$ "--- действие группы на множестве. Тогда
  \emph{ядром} этого действия называется $\Ker \phi = \{ g \in G \mid
  \Forall{\omega \in \Omega} g(\omega) = \omega \}$.
  Действие называется  \emph{точным} (или \emph{эффективным}), если
  $\Ker \phi = \{ e \}$, т. е. если $\phi$ "--- мономорфизм.
\end{definition}

\begin{examples}~
  \begin{enumerate}
    \item $G = GL(V)$ действует на $V$:
      $\phi \in GL(V)$, $v \in V$,
      то $\phi(v) = \phi(v)$ (точно).
    \item $S_n$ действует на $\{ 1, \dots, n \}$
      тривиальным образом (точно).
    \item Если $G$ действует на $\Omega$,
      то можно определить действие $G$ на $\Omega^2$:
      $g((\omega_1, \omega_2)) = (g(\omega_1), g(\omega_2))$.
    \item Рассмотрим группу $O_2$ "--- всех преобразований плоскости, и в
      $R^2$ рассмотрим правильный $n$-угольник $D_n$ с центром в $O$ Тогда можно
      определить группу диэдра $D_n = \{ \phi \in O_2 \mid \phi(D_n) = D_n \}$.
      Тогда $D_n$ действует на:
      \begin{enumerate}
	\item Плоскость $R^2$
	\item Вершины $n$-угольника
	\item Стороны $n$-угольника
	\item Множество точке $n$-угольника
      \end{enumerate}
  \end{enumerate}
\end{examples}

\begin{definition}
  Пусть $G$ действует на $\Omega$, $\omega \in \Omega$
  Тогда \emph{орбитой} элемента $\omega$
  называется
  $G(\omega) = \{ g(\omega) \mid g \in G \}$.
  Два элемента $\omega_1, \omega_2 \in \Omega$
  \emph{эквивалентны} относительно действия,
  если $\omega_2 \in G(\omega_1)$.
\end{definition}

\begin{proposition}
  Определённое отношение является отношением эквивалентности, его классы "---
  это орбиты действия.
\end{proposition}
\begin{proof}
  $\omega_1 = e(\omega_1)$, т. е. отношение рефлексивно.
  Если $\omega_2 \in G(\omega_1)$,
  то $\omega_2 = g(\omega_1) \To
  g^{-1}(\omega_2) = g^{-1}(g(\omega_1)) =
  e(\omega_1) = \omega_1 \To \omega_1 \in G(\omega_2)$ "---
  отношение симметрично. Если $\omega_2 = g_1(\omega_1)$, $\omega_3 =
  g_2(\omega_2)$, то $\omega_3 = g_2(g_1(\omega_1)) = (g_2 \circ g_1)
  (\omega_1)$ "--- транзитивность.

  Наконец, $\{ \omega' \mid \omega' \in G(\omega)\} = G(\omega)$.
\end{proof}
\begin{corollary}
  $\Omega$ разбивается на орбиты действия.
\end{corollary}

\begin{definition}
  Действие называется \emph{транзитивным}, если у него одна орбита,
  т. е. $\Forall{\omega, \omega' \in \Omega} \Exists{g \in G}
  \omega' = g(\omega)$.
\end{definition}

\begin{example}
  У действия $GL(V)$ на $V$ две орбиты: $\{ 0 \}$ и $V \setminus \{0\}$.
\end{example}

% --- 2016-09-29 (или 24) ---
\begin{definition}
  Пусть $G$ действует на $\Omega$, $\omega \in \Omega$.
  Тогда $\St(\omega) =
  \{ g \in G \mid g(\omega) = \omega \}$
  называется \emph{стабилизатором}
  (\emph{стационарной подгруппой})
  элемента $\omega$.
\end{definition}

\begin{proposition}
  $\St(\omega) < G$
\end{proposition}
\begin{proof}
  Если $g, h \in St(\omega)$,
  то $(gh)(\omega) = g(h(\omega)) =
  \omega \To gh \in \St(\omega)$.
  $g^{-1}(\omega) = g^{-1}(g(\omega)) = e(\omega) = \omega
  \To g^{-1} \in \St(\omega)$.
\end{proof}

\begin{proposition}
  Пусть $G$ действует на $\Omega$,
  $\omega \in \Omega$,
  $\omega' \in G(\omega)$;
  пусть $\omega' = g'(\omega)$.
  Тогда $\{ g \in G \mid \omega' = g(\omega) \} = g' St(\omega) =
  St(\omega') g'$
\end{proposition}
\begin{proof}
  $\omega' = g(\omega) \oTTo g'^{-1}(\omega') = (g'^{-1}g)(\omega) \oTTo
  \omega = (g'^{-1}g)(\omega) \oTTo g'^{-1}g \in St(\omega) \oTTo
  g \in g'St(\omega)$. Наоборот, $\omega' = g(\omega) \oTTo \omega = g^{-1}(\omega')
  \oTTo g^{-1} \in g'^{-1} \St(\omega') \oTTo g \in \St(\omega') g'$.
\end{proof}
\begin{corollary}
  $\St(\omega) = g'^{-1} St(\omega') g'$,
  т. е. стабилизаторы двух элементов одной
  орбиты сопряжены.
\end{corollary}

\begin{corollary}
  $|G(\omega)| = |G : St(\omega)|$
  (если $G(\omega)$ "--- конечна,
  то $|G(\omega)| = \frac{|G|}{|St(\omega)|}$).
\end{corollary}
\begin{proof}
  Сопоставим любому $\omega' \in G(\omega)$
  множество $\{ g \in G \mid \omega' = g(\omega) \}$.
  Это множество "---
  левый смежный класс по $\St(\omega)$.
  Ясно, что разным элементам в $G(\omega)$
  соответствуют разные смежные классы и
  любому смежному классу соответствует 
  $\omega' \in G(\omega)$.
  Итак, мы придумали биекцию
  между $G(\omega)$ и $\Factor{G}[\St(\omega)]$.
\end{proof}

\begin{theorem}[формула орбит]
  Пусть группа $G$ действует
  на множество $\Omega = \Omega_1 \sqcup \dots \sqcup \Omega_k$,
  где $\Omega_i$ "--- орбиты действия,
  пусть $\omega_i \in \Omega_i$.
  Тогда
  $$|\Omega| = \sum_{i = 1}^k |\Omega_i| = \sum_{i = 1}^k |G : \St(\omega_i)|$$
\end{theorem}

\subsection{Примеры действия группы}
\subsubsection{Действие на себя левыми сдвигами}
$\Omega = G$, $\Forall{g, \omega \in G}
g(\omega) = g \cdot \omega$.
Орбиты -- вся $G$ (т. е. действие транзитивно),
ядро тривиально
($g \ne e \To g \omega \ne \omega$;
такое действие называется \emph{свободным}).
$\St(\omega) = e$.
Это значит, что верна теорема Кэли:
\begin{theorem}[Кэли]
  Пусть $G$ "--- группа.
  Тогда в $S(G)$ есть подгруппа $H \cong G$.
\end{theorem}
\begin{proof}
  Действие левыми сдвигами определяет гомоморфизм $\phi : G \to S(\Omega) = S(G)$,
  причём $\Ker \phi = \{e\}$. Значит, $\phi$ "--- мономорфизм,
  $G \cong \Img \phi < S(G)$.
\end{proof}

\subsubsection{Действие на смежные классы сдвигами}
Пусть $H < G$.
Тогда $G$ действует левыми сдвигами на $\Omega = \Factor{G}[H]$:
$\Forall{g, x \in G} g(xH) = gxH$
($g_1(g_2(xH)) = g_1g_2xH = (g_1g_2)(xH$).
Орбита "--- $\Factor{G}[H]$.
Стабилизатор $xH$ "--- это $\{g \mid gxH = xH\}$.
Но $gxH = xH \oTTo gx \in xH \oTTo g \in xHx^{-1}$.
Значит, ядро действия есть
$$K = \bigcap_{x \in G} x H x^{-1} = \bigcap_{x \in G} \St(xH).$$
\begin{proposition}
  $K$ "--- наибольшая по включению подгруппа в $H$, которая нормальна в $G$.
\end{proposition}
\begin{proof}
  $K \NSG G$, т. к. это "--- ядро действия.
  Наоборот, если $K' \NSG G$, $K' < H$,
  то $\Forall{x \in G} K' = xK'x^{-1} < xHx^{-1} \To
  K' < \bigcap_{x \in G} xHx^{-1} = K$
\end{proof}

\begin{exercise}
  Пусть $H < G$, $|G:H| = n$.
  Тогда существует $L \NSG G$ такая,
  что $|G:L| \divides n!$.
\end{exercise}

Аналогичные действия правыми сдвигами определяются так:
$g(\omega) = \omega g^{-1}$.
$g(h(\omega)) = \omega h^{-1} g^{-1} = \omega (gh)^{-1} = (gh)(\omega)$,
$e(\omega) = \omega$.

\subsubsection{Действие сопряжением на себя}
$\Omega = G$, $g(\omega) = \omega^{g^{-1}}$.
Проверка: $g(h(\omega)) = (\omega^{h^{-1}})^{g^{-1}} =
\omega^{h^{-1}g^{-1}} = \omega^{(gh)^{-1}} = (gh)(\omega)$,
$e(\omega) = \omega$.
Орбита $G(\omega) = \omega^G$ "--- класс сопряжённости,
стабилизатор: $g\omega g^{-1} = \omega \oTTo g \omega = \omega g$,
т. е. $\St(\omega) = \{ g \in G \mid g\omega = \omega g \}$ "---
\emph{централизатор} элемента $\omega$,
обозначается $C_G(\omega)$.
Разумеется, $C_G(\omega)$ "--- наибольшая по включению подгруппа,
все элементы которой перестановочны с $\omega$.
Ядро действия есть
$$\bigcap_{\omega \in G} C_G(\omega) =
\{ g \in G \mid \Forall{\omega \in G} g \omega = \omega g \} = Z(G).$$
Это множество называется \emph{центром} группы.

\begin{remark}
  $Z(G) \NSG G$.
\end{remark}

\begin{proposition}
  Если $G$ "--- конечная группа, $\omega \in G$,
  то $|\omega^G| \divides \frac{|G|}{|\omega|}$.
\end{proposition}
\begin{proof}
  $$|\omega^G| = |G(\omega)| = \frac{|G|}{|\St(\omega)|} =
  \frac{|G|}{|C_G(\omega)|}.$$
  Однако, $\omega \in C_G(\omega) \To
  \langle \omega \rangle < C_G(\omega) \To
  |\omega| = |\langle \omega \rangle | \divides |C_G(\omega)|$.
  Значит, $|\omega^G| = \frac{|G|}{|C_G(\omega)|}
  \divides \frac{|G|}{|\omega|}$
  Равенство достигается,
  если $C_G(\omega)$ тривиален (степени $\omega$).
\end{proof}

\begin{definition}
  \emph{Автоморфизм} группы $G$ "---
  это изоморфизм $G \to G$.
  Множество всех автоморфизмов $G$
  обозначается $\Aut(G)$, это "---
  группа относительно композиции.

  Автоморфизм $\phi \in \Aut(G)$ называется \emph{внутренним},
  если $\Exists{h \in G} \Forall{g \in G} \phi(g) = g^h$.
  Множество всех внутренних автоморфизмов
  обозначается через $\Inn(G)$.
\end{definition}

\begin{proposition}
  $\Inn(G) < \Aut(G)$. Более того, $\Inn(G) \cong \Factor{G}[Z(G)]$.
\end{proposition}
\begin{proof}
  Рассмотрим действие $G$ на себя сопряжениями. Это "--- гомоморфизм
  $I : G \to S(G)$, образ элемента $g \in G$ обозначим $I_g$.
  Тогда $\Forall{g \in G} I_g \in \Aut(G)$: $I_g(xy) = (xy)^{g^{-1}}
  = x^{g^{-1}} y^{g^{-1}} = I_g(x) I_g(y)$
  (кроме того, $I_g$ "--- биекция).
  Значит, $I : G \to \Aut(G) < S(G)$,
  $\Inn(G) = \Img I < \Aut(G)$,
  $\Img I \cong \Factor{G}[\Ker I] = \Factor{G}[Z(G)]$.
\end{proof}

\begin{exercise}
  \( \Inn(G) \NSG \Aut(G) \)
\end{exercise}

\subsubsection{Действие сопряжением на подгруппы}
\( \Omega \) "--- множество подгрупп \( G \),
$g(H) = H^{g^{-1}} = g H g^{-1} < G$.
Орбита подгруппы $H$ "---
все подгруппы, сопряжённые с $H$.
Стабилизатор $H$ $\St(H) = \{ g \in G \mid gH = Hg \} = N_G(H)$ "---
\emph{нормализатор} подгруппы $H$.
$N_G(H)$ "--- наибольшая (по включению) подгруппа в $G$,
в которой $H$ нормальная.

\begin{remark}
Количество подгрупп, сопряжённых с $H$ есть $|G:N_G(H)|$.
\end{remark}

\end{document}

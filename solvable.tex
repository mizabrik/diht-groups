\documentclass[main]{subfiles}

\begin{document}
\section{Разрешимые группы}
\begin{definition}
  Группа $G$ называется \emph{разрешимой},
  если существует такое $n \in \Natural$,
  что $G^{(n)} = \{ e \}$.
\end{definition}

\begin{example}
  $S'_3 = A_3 = \langle \Cycle{1&2&3} \rangle$, $A_3$ "--- абелева, а потому $A_3' = \{ e \}
  = S_3^{(2)}$. Значит, $S_3$ "--- разрешима.
\end{example}

\begin{remark}
  Наименьшее $n$ такое, что $G^{(n)} = \{e\}$, называется \emph{ступенью разрешимости}.
\end{remark}

\begin{theorem}
  Пусть $K \NSG G$. Тогда $G$ разрешима $\oTTo$ $K$ и $\Factor{G}[K]$ разрешимы.
\end{theorem}
\begin{itemproof}
  \item[$\To$] $K < G \To K^{(n)} < G^{(n)}$. Значит, $K$ разрешимо.
    Пусть $\pi : G \to \Factor{G}[K]$ "--- канонический эпиморфизм. Тогда
    $(\Factor{G}[K])' = \pi(G')$, и по индукции $(\Factor{G}[K])^{(n)} = \pi(G^{(n)})$.
    Т. к. $G^{(n)} = \{ e \}$ при некотором $n$, $(\Factor{G}[K])^{(n)} = \{ K \}$ $\To$
    $\Factor{G}[K]$ разрешима.
  \item[$\oT$] Пусть $K^{(n)} = \{ e \}$, $(\Factor{G}[K])^{(l)} = \{ e \}$.
    Тогда $\pi(G^{(l)}) = (\Factor{G}[K])^{(l)} = \{e \}$,
    т. е. $G^{(l)} < \Ker \pi = K$.
    Значит, $G^{(l+n)} = (G^{(l)})^{(n)} < K^{(n)} = \{ e \}$.
\end{itemproof}

\begin{corollary}
  Пусть $K_1, K_2 \NSG G$ "--- разрешимые (нормальные) подгруппы. Тогда
  $K_1 \cdot K_2 \NSG G$ также разрешима.
\end{corollary}
\begin{proof}
  Заметим, что $K_1 \NSG K_1 \cdot K_2$; $K_1$ "--- разрешима по условию,
  $\Factor{(K_1 \cdot K_2)}[K_1] \cong
  \Factor{K_2}[(K_1 \cap K_2)]$
  (по первой теореме об изоморфизме) "---
  также разрешима, т. к. разрешимы $K_1$
  и \( K_2 \) (по теореме).
  Значит, $K_1 \cdot K_2$ также разрешима.
\end{proof}

\begin{corollary}
  В любой конечной группе $G$ существует
  наибольшая по включению нормальная разрешимая подгруппа
  (это просто произведение всех нормальных разрешимых подгрупп).
\end{corollary}

\begin{theorem}
  Пусть $G$ "--- группа. Тогда равносильны следующие утверждения:
  \begin{enumerate}
    \item $G$ "--- разрешима.
    \item Существует цепочка подгрупп $G = G_0 > G_1 > \dots > G_k = \{ e \}$
      такая, что $G_i \NSG G$ и
      $\Factor{G_i}[G_{i+1}]$ "--- абелева.
    \item Существует цепочка подгрупп $G = G_0 > G_1 > \dots > G_k = \{ e \}$ такая,
      что $G_{i + 1} \NSG G_i$
      и $\Factor{G_i}[G_{i+1}]$ "--- абелева.
  \end{enumerate}
\end{theorem}
\begin{itemproof}
  \item[$1 \To 2$] Положим $G_i = G^{(i)}$.
    Т. к. $G$ "--- разрешима,
    $G^{(k)} = \{ e \}$ при некотором $k$.
    Уже доказано, что $G^{(i)} \NSG G$ и
    $\Factor{G^{(i)}}[G^{(i + 1)}] = \Factor{G^{(i)}}[(G^{(i)})']$ "--- абелева.
  \item[$2 \To 3$] Тривиально.
  \item[$3 \To 1$] Покажем, что $G^{(i)} < G_i$.
    При $i = 0$ это верно.
    Пусть $G^{(i)} < G_i$,
    тогда рассмотрим канонический эпиморфизм
    $\pi_i : G_i \to \Factor{G_i}[G_{i+1}]$.
    $\pi_i(G^{(i)}) < \Factor{G_i}[G_{i + 1}]$,
    т. е.  $\pi_i(G^{(i)})$ "--- абелева группа.
    Это значит, что у гомоморфизма
    $\pi_i \bigr|_{G^{(i)}}$
    ядро содержит $(G^{(i)})' = G^{(i+1)}$,
    т. е. $G^{(i + 1)} < \Ker \pi_i \cap G^{(i)}
    < \Ker \pi_i = G_{i + 1}$.
    Итак, $G^{(k)} < G_k = \{ e \} \To G^{(k)} = \{ e \}$,
    т. е. $G$ разрешима.
\end{itemproof}

\begin{remark}
  Цепочка в $(2)$ называется \emph{нормальным рядом подгрупп с абелевыми факторами}.
  Цепочка в $(3)$ называется \emph{субнормальным рядом подгрупп с абелевыми факторами}.
\end{remark}

\begin{proposition}
  Любая $p$-группа разрешима.
\end{proposition}
\begin{proof}
  Пусть $G$ "--- $p$-группа, $|G| = p^n$. Докажем, что $G$ разрешима индукцией по $n$.
  При $n = 1$ $G$ "--- циклическая $\To$ абелева $\To$ $G' = \{ e \}$.
  Пусть $n > 1$. Положим $Z = Z(G) \ne \{ e \}$. Если $Z = G$, то $G$ абелева, а
  потому разрешима. Иначе $|Z| = p^k$, $|\Factor{G}[Z]| = p^{n - k}$, где $1 \le k \le n - 1$.
  Значит, $Z$ и $\Factor{G}[Z]$ разрешимы по предположению индукции, а по теореме разрешима
  и $G$.
\end{proof}

\begin{remark}
  Из этого доказательства можно получить нормальный ряд подгрупп с абелевыми
  факторами. Положим $H_0 = \{ e \}$, $H_1 = Z(G)$;
  $H_2 = \pi^{-1}(Z(\Factor{G}[H_1]))$,
  где $\pi : G \to \Factor{G}[H_1]$ "--- канонический эпиморфизм.
  Аналогично строятся \( H_3, \dots, H_k \).
  Тогда $\{ e \} = H_0 < H_1 < \dots < H_k = G$ "---
  требуемый нормальный ряд.
\end{remark}

\begin{theorem}
  Пусть $G$ "--- $p$-группа, $|G| = p^n$. Тогда для любого $0 \le k \le n-1$
  $\Exists{H \NSG G} |H| = p^k$.
\end{theorem}
\begin{proof}
  Индукция по $k$. Если $k = 0$, то $H = \{ e \}$.
  Пусть $k > 0$. Обозначим $Z = Z(G) \ne \{ e \}$.
  Если $e \ne g \in Z$, то $\ord g = p^l$. Положим $h =
  g^{p^{l - 1}}$; тогда $\ord h = p$. Пусть $H = \langle h \rangle \To
  |H| = p.$ Более того, $H < Z \To H \NSG G$. В группе $\Factor{G}[H]$
  по предположению индукции найдётся нормальная подгруппа порядка
  $p^{k-1}$; по второй теореме об изоморфизме эта подгруппа имеет
  вид $\Factor{K}[H]$, где $H < K \NSG G$. Итак, $K \NSG G$ и $|K| = |\Factor{K}[H]| \cdot |H| =
  p^{k-1} \cdot p = p^k$.
\end{proof}

\begin{remark}
  Т. к. любая $p$-группа $G$ разрешима, $G' \ne G$.
\end{remark}

\begin{exercise}
  Докажите, что $S_4$ разрешима.
\end{exercise}

\end{document}
